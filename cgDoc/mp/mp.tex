%-----------------------------------------------------------------------
% Cgmp: The multi-physics solver
%        USER GUIDE
%-----------------------------------------------------------------------
\documentclass[10pt]{article}
% \usepackage[bookmarks=true]{hyperref}  % this changes the page location !
\usepackage[bookmarks=true,colorlinks=true,linkcolor=blue]{hyperref}

% \input documentationPageSize.tex
\hbadness=10000 
\sloppy \hfuzz=30pt

% \voffset=-.25truein
% \hoffset=-1.25truein
% \setlength{\textwidth}{7in}      % page width
% \setlength{\textheight}{9.5in}    % page height

\usepackage[usenames,dvipsnames,svgnames,table]{xcolor}

\usepackage{calc}
\usepackage[lmargin=.75in,rmargin=.75in,tmargin=.75in,bmargin=.75in]{geometry}

% \input homeHenshaw

% \input{pstricks}\input{pst-node}
% \input{colours}
\newcommand{\blue}{\color{blue}}
\newcommand{\green}{\color{green}}
\newcommand{\red}{\color{red}}
\newcommand{\black}{\color{black}}



% \documentclass[11pt]{article}
% \usepackage{times}  % for embeddable fonts, Also use: dvips -P pdf -G0

% \input documentationPageSize.tex

% \input homeHenshaw

% \input{pstricks}\input{pst-node}
% \input{colours}

\usepackage{amsmath}
\usepackage{amssymb}

\usepackage{verbatim}
\usepackage{moreverb}

\usepackage{graphics}    
\usepackage{epsfig}    
\usepackage{calc}
\usepackage{ifthen}
\usepackage{float}
% the next one cause the table of contents to disappear!
% * \usepackage{fancybox}

\usepackage{makeidx} % index
\makeindex
\newcommand{\Index}[1]{#1\index{#1}}

\usepackage{tikz}
\usepackage{pgfplots}
\usepackage{xargs}% for optional args to \newcommandx
% define the trimming commands:
\input ../common/trimFig.tex
\input ../common/plotFigureMacros


% % ---- we have lemmas and theorems in this paper ----
% \newtheorem{assumption}{Assumption}
% \newtheorem{definition}{Definition}

% % \newcommand{\homeHenshaw}{/home/henshaw.0}

% \newcommand{\Overture}{{\bf Over\-ture\ }}
% \newcommand{\ogenDir}{\homeHenshaw/Overture/ogen}

% \newcommand{\cgDoc}{\homeHenshaw/cgDoc}
% \newcommand{\vpDir}{\homeHenshaw/cgDoc/ins/viscoPlastic}

% \newcommand{\obFigures}{\homeHenshaw/res/OverBlown/docFigures}  % for figures
% \newcommand{\convDir}{.}

\input ../common/defs
% \input ../common/wdhDefinitions.tex

% \def\comma  {~~~,~~}
% \newcommand{\uvd}{\mathbf{U}}
% \def\ud     {{    U}}
% \def\pd     {{    P}}
% \def\calo{{\cal O}}

% \newcommand{\mbar}{\bar{m}}
% \newcommand{\Rbar}{\bar{R}}
% \newcommand{\Ru}{R_u}         % universal gas constant
% % \newcommand{\Iv}{{\bf I}}
% % \newcommand{\qv}{{\bf q}}
% \newcommand{\Div}{\grad\cdot}
% \newcommand{\tauv}{\boldsymbol{\tau}}
% \newcommand{\thetav}{\boldsymbol{\theta}}
% % \newcommand{\omegav}{\mathbf{\omega}}
% % \newcommand{\Omegav}{\mathbf{\Omega}}

% \newcommand{\Omegav}{\boldsymbol{\Omega}}
% \newcommand{\omegav}{\boldsymbol{\omega}}
% \newcommand{\sigmav}{\boldsymbol{\sigma}}
% \newcommand{\cm}{{\rm cm}}
% \newcommand{\Jc}{{\mathcal J}}

% \newcommand{\sumi}{\sum_{i=1}^n}
% % \newcommand{\half}{{1\over2}}
% \newcommand{\dt}{{\Delta t}}

% \def\ff {\tt} % font for fortran variables

% % define the clipFig commands:
% \input clipFig.tex

% \newcommand{\bogus}[1]{}  % removes is argument completely



% ------------------------------------------- BEGIN DOCUMENT ---------------------------
\begin{document}



\vspace{5\baselineskip}
\begin{flushleft}
{\Large
{\bf Cgmp}: A Multi-Physics Multi-Domain Solver for FSI and CHT \\
    User Guide and Reference Manual \\
}
\vspace{2\baselineskip}
William D. Henshaw,\\
Department of Mathematical Sciences, \\
Rensselaer Polytechnic Institute, \\
Troy, NY, USA, 12180.

% Kyle K. Chand  \\
% William D. Henshaw  \\
% Centre for Applied Scientific Computing  \\
% Lawrence Livermore National Laboratory      \\
% Livermore, CA, 94551.  \\
% henshaw@llnl.gov \\
% % http://www.llnl.gov/casc/people/henshaw \\
% http://www.llnl.gov/casc/Overture\\
\vspace{\baselineskip}
\today\\
\vspace{\baselineskip}
% UCRL-MA-134289

\vspace{4\baselineskip}

\noindent{\bf\large Abstract:}

This document describes {\bf Cgmp}, a solver written using the \Overture framework 
to solve multi-physics multi-domain problems. The solver can be used, for example,
to solve conjugate heat transfer (CHT) problems where fluid flow in one domain is coupled to heat transfer
in another {\em solid} domain. CgMp can also be used to solve fluid-structure interaction (FSI) problems
such as deforming solids in an incompressible fluid.

\end{flushleft}

\clearpage
\tableofcontents
\listoffigures

\vfill\eject


\section{Introduction}

This document is currently under development. 


\bigskip
\input tex/cgmpIntro 

% Cgmp solves problems on overset (overlapping/composite/Chimera) grids and is built upon the \Overture 
% framework~\cite{Brown97},\cite{Henshaw96a},\cite{iscope97}. 
% A block diagram of the main components of Overture is given in Figure~\ref{fig:OvertureBlockDiagram}.
% Overture provides support for solving PDEs on overset grids.
% CgMp is primarily a driver program that calls different CG (Composite Grid) solvers: (see Figure~\ref{fig:CgMpBlockDiagram}),
% \begin{description}
%    \item[\quad CgAd:] solves for heat transfer in solids.
%    \item[\quad CgCns:] solves compressible flow (reacting, multi-fluid, multi-phase). 
%    \item[\quad Cgins:] solves for incompressible flow and heat transfer. 
%    \item[\quad CgSm:] solves the equations of linear and nonlinear elasticity for the deformation solids.
% \end{description}
% CgMp also coordinates the transfer of information between different domain solvers (e.g. the temperature and 
% fluxes at the interface between two heat conducting materials.)
% CgMp can join any number of domain solvers. There can be multiple instances of CgAd/CgCns/CgIns/CgSm solvers, for example, 
% for different types of fluid and solids.

% \input tex/OvertureBlockDiagram

% \input tex/cgmpBlockDiagram

% ---------------------------------------------------------------------------------------------------------
\section{The Equations}


\clearpage
\input timeStepping


\clearpage
\input results


% ==============================================================================================================
\clearpage
\input examples


% -------------------------------------------------------------------------------------------------
\vfill\eject
% -------------------------------------------------------------------------------------------------
\bibliography{../common/journalISI,../common/henshaw,../common/henshawPapers}
\bibliographystyle{siam}



\printindex


\end{document}
