%
% Overlapping grid cartoon. 
%   NOTE:  Use xolor options:  \documentclass[xcolor=rgb,svgnames,dvipsnames]{article}
%
\begin{figure}[hbt]
\begin{center}
%
\begin{tikzpicture}[scale=1]
\useasboundingbox (.75,.75) rectangle (15.5,5.75);  % set the bounding box (so we have less surrounding white space)
% 
% solid grid:
\begin{scope}[xshift=1cm,yshift=1cm]
%
%
\fill[black!10!white,xshift=.5cm,yshift=.5cm] (0,0) -- (2.583333,0) arc (180:90:1.416667) -- (4.,4.) -- (0,4.) -- (0,0);
%
\draw[-,thick,blue,yshift=.0 cm] 
   \foreach \x/\y in {1.5/0,1.5/.5,2/1,2/1.5,2.5/2,3/2.5,4/3,5/3.5,5/4,5/4.5,5/5}{ (0,\y) -- (\x,\y) }
   \foreach \x/\y in {0/0,.5/0,1/0,1.5/0,2/1,2.5/2,3/2.5,3.5/3,4/3,4.5/3.5,5/3.5}{ (\x,\y) -- (\x,5) };
  % -- annulus : lines
  \begin{scope}[xshift=4.5cm,yshift=0.5cm]
    % \draw[thick,green] \foreach \r in {1,1.5,2,2.5,3,3.5}{ (0,\r) arc (90:190:\r)  (0,\r) arc (90:80:\r) };
    \draw[thick,green] \foreach \r in {1.000000,1.416667,1.833333,2.250000,2.666667,3.083333,3.500000}{ (0,\r) arc (90:190:\r)  (0,\r) arc (90:80:\r) };
    % radial lines
    % \draw[thick,green] ( 0,1) -- ( 0,3.5);
    % \draw[thick,green] (-1,0) -- (-3.5,0);
    % Note: these values from overlappingGridCartoon.m
    \draw[thick,green]
     (0.173648,0.984808)  -- (0.607769,3.446827)
     (0.000000,1.000000)  -- (0.000000,3.500000)
     (-0.173648,0.984808) -- (-0.607769,3.446827)
     (-0.342020,0.939693) -- (-1.197071,3.288924)
     (-0.500000,0.866025) -- (-1.750000,3.031089)
     (-0.642788,0.766044) -- (-2.249757,2.681156)
     (-0.766044,0.642788) -- (-2.681156,2.249757)
     (-0.866025,0.500000) -- (-3.031089,1.750000)
     (-0.939693,0.342020) -- (-3.288924,1.197071)
     (-0.984808,0.173648) -- (-3.446827,0.607769)
     (-1.000000,0.000000) -- (-3.500000,0.000000)
     (-0.984808,-0.173648) -- (-3.446827,-0.607769);
% 
  \end{scope}
  % boundary in red 
  \draw[very thick,red,xshift=.5cm,yshift=.5cm] (0,0) -- (2.583333,0) arc (180:90:1.416667) -- (4.,4.) -- (0,4.) -- (0,0);
%
%  square interp. pts
   \filldraw[green] (1.5,.5)  circle (3pt)
                 (1.5,1 )  circle (3pt)
                 (2  ,1 )  circle (3pt)
                 (2 ,1.5)  circle (3pt)
                 (2 , 2 )  circle (3pt)
                 (2.5,2 )  circle (3pt)
                 (2.5,2.5) circle (3pt)
                 (3 , 2.5) circle (3pt)
                 (3  ,3 )  circle (3pt)
                 (3.5,3 )  circle (3pt)
                 (4  ,3. ) circle (3pt)
                 (4  ,3.5) circle (3pt)
                 (4.5,3.5) circle (3pt);
%
% --- annulus interp. pts
  \begin{scope}[xshift=4.5cm,yshift=0.5cm]
      \filldraw[blue]
       (0.000000,3.500000)    circle (3pt)
       (-0.607769,3.446827)   circle (3pt)
       (-1.197071,3.288924)  circle (3pt) 
       (-1.750000,3.031089)  circle (3pt) 
       (-2.249757,2.681156)  circle (3pt) 
       (-2.681156,2.249757)  circle (3pt) 
       (-3.031089,1.750000)  circle (3pt) 
       (-3.288924,1.197071)  circle (3pt) 
       (-3.446827,0.607769)  circle (3pt) 
       (-3.500000,0.000000)  circle (3pt);
% 
  \end{scope}
%   --------- overlapping grid labels -----------
%   \draw (1,4) node[anchor=north west,draw,fill=white] {\large$\Omega$};
   \draw (1.25,3.5) node[thick,draw=blue,fill=white] {\large$G_1$};
   \draw (3.25,1.95) node[thick,draw=green,fill=white] {\large$G_2$};
\end{scope}
%
% -------------------------------------------------------------------
% ----------------------- Square unit Square ------------------------
% -------------------------------------------------------------------
% \definecolor{ghostColour}{named}{Aquamarine}
% \definecolor{ghostColour}{named}{Emerald}
% \definecolor{ghostColour}{named}{Goldenrod}
% \definecolor{ghostColour}{named}{SeaGreen}
\definecolor{ghostColour}{named}{DodgerBlue}
% define triangles for ghost points %  cos(pi/3)=sqrt(3)/2 = .8660   .2*.866 = .173  .3*.866 = .26
\newcommand{\mytrix}{(\x,-.15) -- ++(.3,0) -- ++(-.15,.26) -- (\x,-.15)}
\newcommand{\mytriy}{(-.15,\y) -- ++(.3,0) -- ++(-.15,.26) -- (-.15,\y)}
%
\begin{scope}[xshift=7cm,yshift=2.25cm,scale=.75]
\draw[-,thick,blue,yshift=.0 cm] 
   \foreach \x in {0,.5,...,5}{ (\x,0) -- (\x,5) }
   \foreach \y in {0,.5,...,5}{ (0,\y) -- (5,\y) };
  % boundary in red 
  \draw[very thick,red,xshift=.5cm,yshift=.5cm] (1.,0) -- (.0,0) -- (.0,4.) -- (4.,4.) -- (4.,3.);
  % 
   \filldraw[green] (1.5,.5)  circle (3pt)
                 (1.5,1 )  circle (3pt)
                 (2  ,1 )  circle (3pt)
                 (2 ,1.5)  circle (3pt)
                 (2 , 2 )  circle (3pt)
                 (2.5,2 )  circle (3pt)
                 (2.5,2.5) circle (3pt)
                 (3 , 2.5) circle (3pt)
                 (3  ,3 )  circle (3pt)
                 (3.5,3 )  circle (3pt)
                 (4  ,3. ) circle (3pt)
                 (4  ,3.5) circle (3pt)
                 (4.5,3.5) circle (3pt);
% holes:
  \filldraw[fill=white,draw=black]  \foreach \x in {2,2.5,...,5}{ (\x,.0) circle (3.5pt) };
  \filldraw[fill=white,draw=black]  \foreach \x in {2,2.5,...,5}{ (\x,.5) circle (3.5pt) };
  \filldraw[fill=white,draw=black]  \foreach \x in {2.5,3,...,5}{ (\x,1.) circle (3.5pt) };
%
  \filldraw[fill=white,draw=black]  \foreach \x in {2.5,3,...,5}{ (\x,1.5) circle (3.5pt) };
  \filldraw[fill=white,draw=black]  \foreach \x in {3,3.5,...,5}{ (\x,2.0) circle (3.5pt) };
  \filldraw[fill=white,draw=black]  \foreach \x in {3.5,4,...,5}{ (\x,2.5) circle (3.5pt) };
  \filldraw[fill=white,draw=black]  \foreach \x in {4.5,5}      { (\x,3.0) circle (3.5pt) };
%  \filldraw[fill=white,draw=black]  \foreach \x in {5}          { (\x,3.5) circle (3.5pt) };
% Ghost points:
  \draw[fill=ghostColour,xshift=-.15cm,yshift=0cm]  \foreach \x in {.5,1.,1.5}{ \mytrix };  
  \draw[fill=ghostColour,xshift=-.15cm,yshift=5cm]  \foreach \x in {.5,1.,...,5}{ \mytrix };  
  \draw[fill=ghostColour,xshift=0cm,yshift=-.15cm]  \foreach \y in {0,.5,...,5}{ \mytriy };
  \draw[fill=ghostColour,xshift=5cm,yshift=-.15cm]  \foreach \y in {3.5,4,4.5}{ \mytriy };
%   --------- labels -----------
   \draw (1.25,3.5) node[thick,draw=blue,fill=white] {\large$G_1$};
\end{scope}
% -------------------------------------------------------------------
% ----------------------- Annulus unit Square ------------------------
% -------------------------------------------------------------------
\begin{scope}[xshift=11.5cm,yshift=2.25cm,scale=.75]
\draw[-,thick,green,yshift=.0 cm] 
   \foreach \x in {0,.454545,...,5}{ (\x,0) -- (\x,5) }
   \foreach \y in {0,.833333,...,5}{ (0,\y) -- (5,\y) };
% boundary in red 
 \draw[very thick,red,xshift=.454545cm,yshift=.833333cm] (0.,4) -- (.0,0) -- (4.0909,0.) -- (4.0909,4);
%
 \filldraw[blue]  \foreach \x in {.454545,.909090,...,4.545454}{ (\x,5) circle (3.5pt) };
% Ghost points:
 \draw[fill=ghostColour,xshift=-.15cm]  \foreach \x in {.454545,.909090,...,4.545454}{ \mytrix };
 \draw[fill=ghostColour,yshift=-.15cm]  \foreach \y in {0,.833333,...,5}{ \mytriy };
 \draw[fill=ghostColour,xshift=5cm,yshift=-.15cm]  \foreach \y in {0,.833333,...,5}{ \mytriy };
%
\end{scope}
% labels
\begin{scope}[xshift=7cm,yshift=.7cm]
  \fill[black!10!white,xshift=-.1cm,yshift=-.25cm] (0,0) -- (3,0) -- (3.,1.3) -- (0,1.3) -- (0,0);
  \filldraw[green,xshift=.0cm,yshift=.8cm] (.25,.0)  circle (3pt);
  \filldraw[blue,xshift=.3cm,yshift=.8cm] (.25,.0)  circle (3pt);
  \draw[xshift=.0cm,yshift=.8cm] (.5,0) node[anchor=west,xshift=6pt] {\small interpolation};
  \draw[fill=ghostColour,xshift=.0cm,yshift=.4cm] (.35,0) \foreach \x in {.1}{ \mytrix } node[anchor=west,xshift=12pt,yshift=3pt] {\small ghost};
  \draw[fill=white,draw=black,xshift=.0cm,yshift=.0cm] (.25,0) circle (3.5pt) node[anchor=west,xshift=6pt] {\small unused};
\end{scope}
%   --------- labels -----------
\begin{scope}[xshift=11.5cm,yshift=2.25cm,scale=.75]
   \draw (1.6,3.27) node[thick,draw=green,fill=white] {\large$G_2$};
\end{scope}
% -------------------------------------------------------------------
% -------------------------------------------------------------------
% -------------------------------------------------------------------
% grid:
%% \draw[step=1cm,gray] (0,0) grid (16,6.);
\end{tikzpicture}
\end{center}
\caption{Left: an overlapping grid consisting of two
structured curvilinear component grids, $\xv=G_1(\rv)$ and $\xv=G_2(\rv)$. Middle and right: 
component grids for the square and annular grids in the unit square parameter space $\rv$. Grid
 points are classified as discretization points, interpolation points or unused points. Ghost points
 are used to apply boundary conditions.}    \label{fig:overlappingGridCartoon}
\end{figure}
