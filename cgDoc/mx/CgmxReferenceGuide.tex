%-----------------------------------------------------------------------
% Reference Guide for CGMX -- Maxwell's Equation Solver
% 
%-----------------------------------------------------------------------
\documentclass{article}
\usepackage[bookmarks=true,colorlinks=true,linkcolor=blue]{hyperref}


% \input documentationPageSize.tex
\hbadness=10000 
\sloppy \hfuzz=30pt

% \voffset=-.25truein
% \hoffset=-1.25truein
% \setlength{\textwidth}{7in}      % page width
% \setlength{\textheight}{9.5in}    % page height

\usepackage{calc}
\usepackage[lmargin=.75in,rmargin=.75in,tmargin=.75in,bmargin=.75in]{geometry}


% \input homeHenshaw

\usepackage{amsmath}
\usepackage{amssymb}

\usepackage{verbatim}
\usepackage{moreverb}

% \usepackage{epsfig}    
% This next section will allow graphics files to be ps or pdf  -- from Jeff via Jeff
% \usepackage{ifpdf}
% \ifpdf
%     \usepackage[pdftex]{graphicx}
%     \usepackage{epstopdf}
%     \pdfcompresslevel=9
%     \pdfpagewidth=8.5 true in
%     \pdfpageheight=11 true in
%     \pdfhorigin=1 true in
%     \pdfvorigin=1.25 true in
% \else
%     \usepackage{graphicx}
% \fi

% \input{pstricks}\input{pst-node}
% \input{colours}

% define the clipFig commands:
\input ../common/trimFig.tex

% --------------------------------------------
\usepackage[usenames]{color} % e.g. \color{red}
\newcommand{\red}{\color{red}}
\newcommand{\blue}{\color{blue}}
\newcommand{\green}{\color{green}}
\newcommand{\jwb}[2]{{\color{red}(old: #1) }{\color{green} #2}}

\usepackage{tikz}


\usepackage{makeidx} % index
\makeindex
\newcommand{\Index}[1]{#1\index{#1}}

\input tex/defs


% ---- we have lemmas and theorems in this paper ----
\newtheorem{assumption}{Assumption}
\newtheorem{definition}{Definition}

% \newcommand{\homeHenshaw}{/home/henshaw.0}

% \newcommand{\primer}{/users/henshaw/res/primer}
% \newcommand{\GF}{/users/\-henshaw/\-res/\-gf}
% \newcommand{\gf}{/users/henshaw/res/gf}
% \newcommand{\mapping}{/users/henshaw/res/mapping}
% 
% \newcommand{\docFigures}{\homeHenshaw/OvertureFigures}
% \newcommand{\figures}{\homeHenshaw/res/OverBlown/docFigures}
% \newcommand{\obFigures}{\homeHenshaw/res/OverBlown/docFigures}  % note: local version for OverBlown
% \newcommand{\maxDoc}{\homeHenshaw/res/maxwell/doc}
% 
% \newcommand{\OVERTUREOVERTURE}{/users/\-henshaw/\-Overture/\-Overture}
% \newcommand{\OvertureOverture}{/users/henshaw/Overture/Overture}

\newcommand{\Overture}{{\bf Overture\ }}
\newcommand{\OverBlown}{{\bf OverBlown\ }}
\newcommand{\overBlown}{{\bf overBlown\ }}

% *** See http://www.eng.cam.ac.uk/help/tpl/textprocessing/squeeze.html
% By default, LaTeX doesn't like to fill more than 0.7 of a text page with tables and graphics, nor does it like too many figures per page. This behaviour can be changed by placing lines like the following before \begin{document}

\renewcommand\floatpagefraction{.9}
\renewcommand\topfraction{.9}
\renewcommand\bottomfraction{.9}
\renewcommand\textfraction{.1}   
\setcounter{totalnumber}{50}
\setcounter{topnumber}{50}
\setcounter{bottomnumber}{50}

\begin{document}


% -----definitions-----
\input ../common/wdhDefinitions.tex

\newlength{\ycbTop}% For colour bar
\newlength{\ycbMid}%

\def\ud     {{    U}}
\def\pd     {{    P}}

\newcommand{\mbar}{\bar{m}}
\newcommand{\Rbar}{\bar{R}}
\newcommand{\Ru}{R_u}         % universal gas constant
% \newcommand{\Iv}{{\bf I}}
% \newcommand{\qv}{{\bf q}}
\newcommand{\Div}{\grad\cdot}
\newcommand{\tauv}{\boldsymbol{\tau}}
\newcommand{\sumi}{\sum_{i=1}^n}
% \newcommand{\half}{{1\over2}}
\newcommand{\dt}{{\Delta t}}
\newcommand{\dx}{{\Delta x}}
\newcommand{\eps}{\epsilon}

\newcommand{\Ic}{{\mathcal I}}
\newcommand{\Lc}{{\mathcal L}}
\newcommand{\Oc}{{\mathcal O}}

\vglue 10\baselineskip
\begin{flushleft}
{\Large
Cgmx Reference Manual: An {Overture} Solver for {Maxwell}'s Equations on Composite Grids\\
}
\vspace{2\baselineskip}
William D. Henshaw  \\
Department of Mathematical Sciences, \\
Rensselaer Polytechnic Institute, \\
Troy, NY, USA, 12180.
% Centre for Applied Scientific Computing  \\
% Lawrence Livermore National Laboratory      \\
% Livermore, CA, 94551.  \\
% henshaw@llnl.gov \\
% http://www.llnl.gov/casc/people/henshaw \\
% http://www.llnl.gov/casc/Overture\\
\vspace{\baselineskip}
\today\\
\vspace{\baselineskip}
% UCRL-MA-134288

\vspace{4\baselineskip}

\noindent{\bf\large Abstract:}

Cgmx is a program that can be used to solve the time-dependent Maxwell's equations
of electromagnetics on composite overlapping grids in two and three space
dimensions. This document is a companion to the Cgmx User Guide. 
This reference guide provides additional details about the equations and approximations used along
with additional examples and numerical results.

\end{flushleft}

\clearpage
\tableofcontents
% \listoffigures


\vfill\eject

\section{Introduction}

   Cgmx is a program that can be used to solve Maxwell's equations of electromagnetics 
on composite overlapping grids~\cite{max2006b}. It is built upon
the \Overture framework~\cite{Brown97},\cite{Henshaw96a},\cite{iscope97}.

For an introduction to Cgmx and its usage see the Cgmx User Guide~\cite{CgmxUserGuide}

More information about
{\bf Overture} can be found on the \Overture home page, {\tt http://www.llnl.gov/\-casc/\-Overture}.

% =======================================================================================================
\clearpage
\section{Method NFDTD} \label{sec:nfdtd}


% =======================================================================================================
\clearpage
\section{Method Yee} \label{sec:nfdtd}



% =======================================================================================================
\clearpage
\section{Accuracy and convergence examples} \label{sec:examples}

In this section we provide some examples that demonstrate the accuracy of the methods.

% -------------------------------------------------------------------------------------------------
\clearpage
\subsection{A comparison of far field boundary conditions for scattering from a PEC cylinder}

In this example we compare the accuracy of different far-field boundary conditions for
the scattering of a plane wave from a PEC cylinder.


\noindent (1) Generate the grid using the command file {\tt cicArg.cmd} in {\tt Overture/sampleGrids}:
\begin{verbatim}
  ogen -noplot cicArg -order=4 -interp=e -factor=4
\end{verbatim}

\noindent (2) Run cgmx with the Engquist-Majda BC {\tt abcEM2} and the PML BC {\tt abcPML}: 
\begin{verbatim}
  cgmx -noplot cic.planeWaveBC -g=cice4.order4.hdf -diss=.5 -ic=zero -rbc=abcEM2 ...
       -tp=1. -tf=100. -show="cylEM2.show"
  cgmx -noplot cic.planeWaveBC -g=cice4.order4.hdf -diss=.5 -ic=zero -rbc=abcPML ...
       -pmlWidth=21 -pmlStrength=50. -tp=1. -tf=100. -show="cylPML.show"
\end{verbatim}

\noindent (3) To plot errors use plotStuff to display the show file. Plot the ``sequence'' with the errors
and then save to a matlab file. 
\begin{verbatim}
plot sequence:errors
  add Ey_error
  add Hz_error
  save results to a matlab file
    scatCyl.m 
\end{verbatim}

{
\begin{figure}[hbt]
\newcommand{\figWidth}{5.5cm}
\newcommand{\trimfig}[2]{\trimFig{#1}{#2}{0.1}{0.05}{.05}{.05}}
\begin{center}
\begin{tikzpicture}[scale=1]
  \useasboundingbox (0,1.25) rectangle (17,5.5);  % set the bounding box (so we have less surrounding white space)
  \draw ( 0.0, 0) node[anchor=south west] {\trimfig{figures/scatCylEM2EyErrT2}{\figWidth}};
  \draw ( 5.7, 0) node[anchor=south west] {\trimfig{figures/scatCylEM2EyErrT4}{\figWidth}};
  \draw (11.4, 0) node[anchor=south west] {\trimfig{figures/scatCylEM2EyErrT6}{\figWidth}};
 % - labels
 %   \draw (\txa,4.75) node[draw,fill=white,anchor=east] {\scriptsize $t=0.5$};
 %   \draw (\txb,4.75) node[draw,fill=white,anchor=east] {\scriptsize $t=1.0$};
 %   \draw (\txc,4.75) node[draw,fill=white,anchor=east] {\scriptsize $t=1.5$};
 %  \draw (current bounding box.south west) rectangle (current bounding box.north east);
% grid:
%  \draw[step=1cm,gray] (0,0) grid (17.0,5);
\end{tikzpicture}
\end{center}
\caption{{\tt abcEM2} : scattering of a plane wave by a PEC cylinder. Errors in $E_y$ at $t=2.$, $4.$ and $6.0$. } 
\label{fig:cyl2dScatEM2}
\end{figure}
}
{
\begin{figure}[hbt]
\newcommand{\figWidth}{5.5cm}
\newcommand{\trimfig}[2]{\trimFig{#1}{#2}{0.1}{0.05}{.05}{.05}}
\begin{center}
\begin{tikzpicture}[scale=1]
  \useasboundingbox (0,1.25) rectangle (17,5.5);  % set the bounding box (so we have less surrounding white space)
  \draw ( 0.0, 0) node[anchor=south west] {\trimfig{figures/scatCylPMLEyErrT2}{\figWidth}};
  \draw ( 5.7, 0) node[anchor=south west] {\trimfig{figures/scatCylPMLEyErrT4}{\figWidth}};
  \draw (11.4, 0) node[anchor=south west] {\trimfig{figures/scatCylPMLEyErrT6}{\figWidth}};
 % - labels
 %   \draw (\txa,4.75) node[draw,fill=white,anchor=east] {\scriptsize $t=0.5$};
 %   \draw (\txb,4.75) node[draw,fill=white,anchor=east] {\scriptsize $t=1.0$};
 %   \draw (\txc,4.75) node[draw,fill=white,anchor=east] {\scriptsize $t=1.5$};
 %  \draw (current bounding box.south west) rectangle (current bounding box.north east);
% grid:
%  \draw[step=1cm,gray] (0,0) grid (17.0,5);
\end{tikzpicture}
\end{center}
\caption{{\tt abcPML} : scattering of a plane wave by a PEC cylinder. Errors in $E_y$ $t=2$, $4$ and $6$. } 
\label{fig:cyl2dScatPML}
\end{figure}
}

{
\begin{figure}[hbt]
\newcommand{\figWidth}{8.cm}
\newcommand{\trimfig}[2]{\trimFig{#1}{#2}{0.}{0.0}{.0}{.0}}
\begin{center}
\begin{tikzpicture}[scale=1]
  \useasboundingbox (0,.75) rectangle (16.5,6.25);  % set the bounding box (so we have less surrounding white space)
  \draw ( 0.0, 0) node[anchor=south west] {\trimfig{figures/scatCylEM2Errors}{\figWidth}};
  \draw ( 8.0, 0) node[anchor=south west] {\trimfig{figures/scatCylPMLErrors}{\figWidth}};
 % - labels
 %   \draw (\txa,4.75) node[draw,fill=white,anchor=east] {\scriptsize $t=0.5$};
 %   \draw (\txb,4.75) node[draw,fill=white,anchor=east] {\scriptsize $t=1.0$};
 %   \draw (\txc,4.75) node[draw,fill=white,anchor=east] {\scriptsize $t=1.5$};
%  \draw (current bounding box.south west) rectangle (current bounding box.north east);
% grid:
%  \draw[step=1cm,gray] (0,0) grid (17.0,5);
\end{tikzpicture}
\end{center}
\caption{Scattering of a plane wave by a PEC cylinder. Errors over time for {\tt abcEM2} (left) and 
   {\tt abcPML} (right). }
\label{fig:cyl2dScatFarFieldComparison}
\end{figure}
}

Figures~\ref{fig:cyl2dScatEM2}-\ref{fig:cyl2dScatPML} shows the errors in $E_y$ for the EM2 and PML 
farfield conditions. Figure~\ref{fig:cyl2dScatFarFieldComparison} shows the maximum errors over
time for the two conditions.

\noindent{\bf Notes:}
\begin{enumerate}
  \item The initial conditions are taken as zero with inhomogeneous boundary conditions on the cylinder boundary.
  \item The errors are initially large as the exact solution develops and propagates outward
   from the cylinder. These initial large errors
   can be seen in the plots at $t=2$.
  \item The EM2 errors show a fast growth for long times, except for the error in $H_z$. The $H_z$ field seems to
          propagate more strongly in the $x$- and $y$- directions (rather than along diagonals) and thus will be
          treated more accurately by the far field conditions.
      The PML errors do not grow as fast but eventually seem to start to grow.
\end{enumerate}







% =======================================================================================================
\clearpage
\section{Boundary Conditions} \label{sec:bc}

% --------------------------------------------------------------------------------
\subsection{Perfect electrical conductor boundary condition}\label{sec:perfectElectricalConductor}

The perfect electrical conductor boundary condition sets the tangential components of the
electric field to zero
\begin{align}
   \tauv_m\cdot\Ev &= 0, \qquad \text{for $\xv \in \partial\Omega_{\rm pec}$}. 
\end{align}
Here $\tauv_m$, $m=1,2$, denote the tangent vectors to the boundary surface, $\partial\Omega_{\rm pec}$.

% --------------------------------------------------------------------------------
\subsection{Engquist-Majda absorbing boundary conditions}\label{sec:EngquistMajdaABC}


The boundary condition {\tt abcEM2} uses the Engquist-Majda absorbing boundary 
condition (defined here of a boundary $x={\rm constant}$), 
\begin{align}
   \partial_t\partial_x u = \alpha \partial_x^2 u + \beta (\partial_y^2+\partial_z^2) u 
\end{align}
With $\alpha=c$ and $\beta=\half c$, this gives a {\em second-order accurate} approximation to 
a pseudo-differential operator that absorbs outgoing traveling waves. 
Here $u$ is any field which satisfies the second-order wave equation.

Notes on the Engquist-Majda absorbing boundary conditions can be found in Appendix~\ref{sec:abc}.


% --------------------------------------------------------------------------------
\subsection{Perfectly matched layer boundary condition}\label{sec:PML}

The boundary condition {\tt abcPML} imposes a perfectly matched layer boundary condition.
With this boundary condition, auxillary equations are solved over a layer (of some number of
specified grid points) next to the boundary. The PML equations we solve
are those suggested in Hagstrom~\cite{Hagstrom1999} and given by (defined here of a boundary $x={\rm constant}$), (*check me*)
\begin{align}
  u_{tt} &= c^2 \Big( \Delta u - \partial_x v - w \Big), \label{eq:PMLa} \\
   v_t &= \sigma(x)( -v + \partial_x u ) , \\
   w_t &= \sigma(x) ( -w  - \partial_x v + \partial_x^2 u ). \label{eq:PMLc}
\end{align}
Here $u$ is any field which satisfies the second-order wave equation and $v$ and $w$ are auxillary variables
that only live in the layer domain. 
The PML damping function $\sigma_1(\xi)$ is given by (Note: scale by $c$ to make dimensionally correct, 2020/03/31) 
\begin{align}
  \sigma(\xi) = c\, a \, \xi^p
\end{align}
where $a$ is the strength, $p$ is the power and where $\xi$ varies from $0$ to $1$ through the layer.



A second-order accurate approximation to equations~\eqref{eq:PMLa}-\eqref{eq:PMLc} is
\begin{align}
  {u_\iv^{n+1} - 2 u_\iv^n + u_\iv^{n-1} \over \dt} &= c^2 \Big( \Delta_{2h} u_\iv - D_{0x} v_\iv  - w_\iv \Big), \label{eq:PMLorder2a}\\
  {v_\iv^{n+1} - v_\iv \over \dt} &= \frac{3}{2} f_\iv^{n} - \half f_\iv^{n-1} , \\
  f_\iv^n  &= \sigma ( -v_\iv^n + D_{0x} u_\iv^n ) , \\
  {w_\iv^{n+1} - w_\iv \over \dt} &= \frac{3}{2} g_\iv^{n} - \half g_\iv^{n-1} , \\
  g_\iv^n  &= \sigma ( -w_\iv^n  - D_{0x} v_\iv^n + D_{+x}D_{-x} u_\iv^n ).   \label{eq:PMLorder2e}
\end{align}
Here we have used a centered in time approximation for the first equation and Adams-Bashforth for the $v$ and $w$ equations.

\noindent{\bf Aside:} Since $\sigma(\xv)$ is large we could use an integrating factor for the $v$ and $w$ equations
and instead solve 
\begin{align}
  \partial_t( e^{\sigma t} v) &= \sigma  e^{\sigma t} \partial_x u  , \\
   \partial_t( e^{\sigma t} v) &= \sigma e^{\sigma t} ( \partial_x v + \partial_x^2 u ). 
\end{align}


A fourth-order accurate approximation to~\eqref{eq:PMLa}-\eqref{eq:PMLc} can be determined using the relations
\begin{align}
  {u_\iv^{n+1} - 2 u_\iv^n + u_\iv^{n-1} \over \dt^2} &= \partial_t^2 u + \frac{\dt^2}{12} \partial_t^4 u + \Oc(\dt^4) \label{eq:PMLorder4a} \\
   &= c^2 \Big( \Delta u - \partial_x v - w \Big)
    + \frac{\dt^2}{12}\Big( \Delta \partial_t^2 u - \partial_t^2\partial_x v - \partial_t^2 w \Big), \\
   &= c^2 \Big( \Delta u - \partial_x v - w \Big)
    + \frac{\dt^2}{12}\Big( c^2 \big( \Delta^2 u  -\partial_x\Delta v - \Delta w\big)  - \partial_t^2\partial_x v - \partial_t^2 w \Big), \\
  v_\iv^{n+1} &= v_\iv^n + \dt\partial_t v + \frac{\dt^2}{2} \partial_t^2 v + \frac{\dt^3}{3!} \partial_t^3 v  
                     + \frac{\dt^4}{4!} \partial_t^4 v + \Oc(\dt^5) \\
  w_\iv^{n+1} &= w_\iv^n + \dt\partial_t w + \frac{\dt^2}{2} \partial_t^2 w + \frac{\dt^3}{3!} \partial_t^3 w  
                     + \frac{\dt^4}{4!} \partial_t^4 w + \Oc(\dt^5)  \label{eq:PMLorder4e}
\end{align}
By taking derivatives of the governing equations ~\eqref{eq:PMLa}-\eqref{eq:PMLc} we can approximate the various
terms in the previous expansions. For example, 
\begin{align}
     \partial_t^2 v &= \sigma(x)( -\partial_t v + \partial_t\partial_x u ) , \\
     \partial_t^3 v &= \sigma(x)( -\partial_t^2 v + \partial_t^2\partial_x u ) , \\
                   &= \sigma(x)\Big( -\partial_t^2 v + c^2 \partial_x (  \Delta u - \partial_x v - w )  \Big) \\
     \partial_t\partial_x v &= \sigma(x)( -\partial_x v + \partial_x^2 u ) + \partial_x\sigma ( - v + \partial_x u ) , \\
     \partial_t^2\partial_x v &= \sigma(x)( -\partial_t\partial_x v + \partial_t\partial_x^2 u )
                                  + \partial_x\sigma ( - \partial_tv + \partial_t\partial_x u ) ,
\end{align}


We obtain second-order approximations for $\partial_t u$, $\partial_t\partial_x u$ and $\partial_t\partial_x^2 u$ from
the follwing approximation to $\partial_t u$, 
\begin{align}
 \partial_t u &= { u_\iv^n - u_\iv^{n-1} \over \dt} - \frac{\dt}{2} \partial_t^2 u + \Oc(\dt^2) \\
\end{align}


% -----------------------------------------------------------------
\subsubsection{PML edge regions} \label{sec:PMLedge}
In the edge region outside an $x$ and $y$ boundary, we solve 
\begin{align}
  u_{tt} &= c^2 \Big( \Delta u - \partial_x v^x - w^x - \partial_y v^y - w^y \Big), \\
   v^x_t &= \sigma(x)( -v^x + \partial_x u ) , \\
   w^x_t &= \sigma(x) ( -w^x  - \partial_x v^x + \partial_x^2 u ), \\
   v^y_t &= \sigma(y)( -v^y + \partial_y u ) , \\
   w^y_t &= \sigma(y) ( -w^y  - \partial_y v^y + \partial_y^2 u ). 
\end{align}
Thus we require 4 auxillary variables in the edge region. 
A second-order accurate approximation can be otained in a similar manner to~\eqref{eq:PMLorder2a}-\eqref{eq:PMLorder2e}.

A fourth-order accurate approximation follows the procedure used to develop~\eqref{eq:PMLorder4a}-\eqref{eq:PMLorder4e}. 
\begin{align}
 & {u_\iv^{n+1} - 2 u_\iv^n + u_\iv^{n-1} \over \dt^2} = \partial_t^2 u + \frac{\dt^2}{12} \partial_t^4 u + \Oc(\dt^4)  \\
   &\quad = c^2 \Big( \Delta u - \partial_x v^x - w^x - \partial_y v^y - w^y \Big)
    + \frac{\dt^2}{12}\Big( \Delta \partial_t^2 u 
                    - \partial_t^2\partial_x v^x - \partial_t^2w^x - \partial_t^2\partial_y v^y - \partial_t^2w^y \Big), \\
   &\quad= c^2 \Big( \Delta u - \partial_x v^x - w^x - \partial_y v^y - w^y\Big) \\
   &\quad + \frac{\dt^2}{12}\Big( c^2 \big( \Delta^2 u - \partial_x\Delta  v^x - \Delta w^x - \partial_y\Delta  v^y - w^y\big)  
                - \partial_t^2\partial_x v^x - \partial_t^2w^x - \partial_t^2\partial_y v^y - \partial_t^2w^y \Big), 
\end{align}


\noindent{\bf NOTE:} Currently the PML edge and corner regions are only approximated to 2nd order for the 4th-order version.

% -----------------------------------------------------------------
\subsubsection{PML corner regions} \label{sec:PMLcorner}

In the corner region outside an $x$, $y$ and $z$ 
boundary we use 6 auxillary variables and solve
\begin{align}
  u_{tt} &= c^2 \Big( \Delta u - \partial_x v^x - w^x - \partial_y v^y - w^y - \partial_z v^z - w^z \Big), \\
   v^x_t &= \sigma(x)( -v^x + \partial_x u ) , \\
   w^x_t &= \sigma(x) ( -w^x  - \partial_x v^x + \partial_x^2 u ), \\
   v^y_t &= \sigma(y)( -v^y + \partial_y u ) , \\
   w^y_t &= \sigma(y) ( -w^y  - \partial_y v^y + \partial_y^2 u ), \\
   v^z_t &= \sigma(z)( -v^z + \partial_z u ) , \\
   w^z_t &= \sigma(z) ( -w^z  - \partial_z v^z + \partial_z^2 u ). 
\end{align}



%-------------------------------------------------------------------------------------------------
\clearpage 
\input tex/interfaceImplementation



%-------------------------------------------------------------------------------------------------
\clearpage 
\input magneticFromElectricField


%-------------------------------------------------------------------------------------------------
\clearpage 
\input numericalResults

% =======================================================================================================
\clearpage
\section{Maxwell's Equations} \label{sec:equations}


The time dependent Maxwell's equations for linear, isotropic and non-dispersive materials are
\begin{align}
  \partial_t \Ev &=  {1\over \eps} \grad\times\Hv - {1\over \eps}\Jv , \label{eq:FOS-Et}  \\
  \partial_t \Hv &= - {1\over \mu} \grad\times\Ev ,  \label{eq:FOS-Ht} \\
  \grad\cdot(\eps\Ev) &=\rho , ~~ \grad\cdot(\mu\Hv) = 0 , \label{eq:FOS-div}
\end{align}
Here $\Ev=\Ev(\xv,t)$ is the electric field, 
$\Hv=\Hv(\xv,t)$ is the magnetic field, $\rho=\rho(\xv,t)$ is the electric charge density,
$\Jv=\Jv(\xv,t)$ is the electric current density,
$\eps=\eps(\xv)$ is the electric permittivity, and $\mu=\mu(\xv)$ is the magnetic permeability.
This first-order system for Maxwell's equations can also be written in a
second-order form. By taking the time derivatives of~(\ref{eq:FOS-Ht}) and
(\ref{eq:FOS-Et}) and using (\ref{eq:FOS-div}) it follows that 
\begin{align}
 \eps\mu~\partial_t^2 \Ev &= \Delta \Ev + \grad\Big( \grad \ln\eps~\cdot\Ev \Big)
        +\grad\ln\mu\times\Big(\grad\times\Ev\Big) 
            -\grad(\frac{1}{\epsilon}\rho)- \mu \partial_t\Jv , \label{eq:waveEGen} \\
 \eps\mu~\partial_t^2 \Hv &= \Delta \Hv + \grad\Big( \grad \ln\mu~\cdot\Hv \Big)
                               +\grad\ln\eps\times\Big(\grad\times\Hv\Big) 
                     + \eps\grad\times(\frac{1}{\epsilon}\Jv ) \label{eq:waveHGen}.
\end{align}
It is evident that the equations for the electric and magnetic field are decoupled with each 
satisfying a vector wave equation with lower order terms.
In the case of constant $\mu$ and $\eps$ and no charges, $\rho=\Jv=0$, 
the equations simplify to the classical second-order wave equations,
\begin{align}
  \partial_t^2 \Ev = c^2~ \Delta \Ev , \qquad
  \partial_t^2 \Hv = c^2~ \Delta \Hv \label{eq:waveE}
\end{align}
where $c^2=1/(\eps\mu)$.
There are some advantages to solving the second-order form of the equations
rather than the first-order system. One advantage is that in some cases it is
only necessary to solve for one of the variables, say $\Ev$. 
If the other variable, $\Hv$ is required, it can be
determined by
integrating equation~\eqref{eq:FOS-Ht} as an ordinary differential equation
with known $\Ev$. Alternatively, as a post-processing step $\Hv$ can be computed from an
elliptic boundary value problem formed by taking the curl of equation~\eqref{eq:FOS-Et}.
Another advantage of the second-order form, which simplifies the implementation on
an overlapping grid, is that there is no need to use a staggered grid formulation. 
Many schemes approximating the first order system~(\ref{eq:FOS-Ht}-\ref{eq:FOS-div}) rely on a
staggered arrangement of the components of $\Ev$ and $\Hv$ such as the
popular Yee scheme~\cite{Yee66} for Cartesian grids. 



\clearpage
\section{Dispersive Maxwell's Equations}

{\red FINISH ME...}


\input tex/interfaceProjection

% ===============================================================================================
\input acknowledgments


% ====================================================================
\appendix
\input abc

\vfill\eject
% \bibliography{\homeHenshaw/papers/henshaw}
\bibliography{../common/henshaw,../common/henshawPapers}
\bibliographystyle{siam}

\printindex

\end{document}

% ***************************************************************************************************




