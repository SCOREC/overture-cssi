%
%  A modified version of this appeas in adegdmi
%
\newcommand{\Kc}{\mathcal{K}}
\newcommand{\dc}{\Kc}% artificial diffusion coeff
\newcommand{\lr}{m}% left right sub-script
\newcommand{\charv}{\psi}%characteristic variable 
\subsection{Interface projection}


The primary interface jump conditions~\eqref{eq:interfaceA} and~\eqref{eq:interfaceB}
are not explicitly imposed when solving for the ghost point values at the interface, only
the second time-derivative of these conditions are imposed. To strictly enforce~\eqref{eq:interfaceA} and~\eqref{eq:interfaceB}
we apply an interface projection at each time-step before assigning the ghost values.



% We first consider a material interface between two non-dispersive materials.
To define the projection 
we consider the solution to a Riemann problem at the interface.
The initial condition for the Riemann problem, taken at a pseudo-time $t=0$, 
consists of a constant left state for $x<0$ and a constant right state for $x>0$. 
The values for these left and right states 
are given by the predicted interface values coming from the left and right interior updates on the interface.
The central state in
the solution to the Riemann problem at $t=0^+$ will define the projected interface state. 
To derive the form of the projected state, it is sufficient to consider the Maxwell's equations in two dimensions
for a TE-z polarized wave, (taking $u=E_x$, $v=E_y$, $w=H_z$), 
\bse
\label{eq:maxwell2d} 
\ba
    \eps_0 \p_t u &= \p_y w -\epsz^{-1} p_t , \\
    \eps_0 \p_tv  & = - \p_x w -\epsz^{-1} q_t , \\
    \mu_0 \p_tw   &= \p_y u - \p_x v , 
\ea
\ese
where $p$ and $q$ are the components of the polarization vector. 
Equations~\eqref{eq:maxwell2d} define a hyperbolic system with lower order terms coming from $p_t$ and $q_t$ which can
be thought of as being proprotional to $u$ and $v$.
Hence we can drop the polarization terms from the characteristic analysis. 
For an interface at $x=0$ we consider waves in the x-direction and the equations then reduce to
\bas
    \eps \p_tu &= 0 , \\
    \eps \p_tv &= - \p_x w, \\
    \mu  \p_tw &= - \p_x v 
\eas
The variable $u$ (normal component of E) is thus a characteristic variable with characteristic speed $\lambda=0$.
The variables $v$ and $w$ form a coupled wave equation with characteristics variables $\charv_\pm$, and
characteristic speeds $\lambda_\pm$, given by 
\bats
 & \charv_+ = v - \eta w, \qquad && \lambda_+ = c, \\
 & \charv_- = v + \eta w, \qquad && \lambda_- = -c, 
\eats
where $\eta$ is the electric impedance, 
\bas
    \eta \eqdef = \sqrt{\f{\mu}{\eps}} .
\eas
Solving the Riemman problem with the interface jump conditions
\bas
&   [\eps u + p]=0, \\
&   [v]=0, \\
&   [w]=0, 
\eas
%- The solution to a Riemann problem with left state $\qv_L$, right state $\qv_R$ and center state $\qv_I$ satisfies
%- \bas
%-  \charv_+ =  v_I - \eta_L w_I = v_L - \eta_L w_L, \\     
%-  \charv_- =  v_I + \eta_R w_I = v_R + \eta_R w_R, 
%- \eas
leads to the central state, denoted by $(v^I,w^I)$, given
\bse
\label{eq:impedanceAverage}
\ba
% v_I &= \f{\eta_R v_L + \eta_L v_R}{\eta_R + \eta_L} + \f{\eta_L\eta_R}{\eta_R + \eta_L}\Big[  w_R - w_L\Big] , \\
v_I &= \f{\eta_L^{-1} \, v_L + \eta_R^{-1} \, v_R}{\eta_R^{-1} + \eta_L^{-1}} + \f{1}{\eta_R^{-1} + \eta_L^{-1}}\Big[  w_R - w_L\Big] , \\
w_I &=\f{\eta_R \, w_R + \eta_L \, w_L}{\eta_R + \eta_L} + \f{1}{\eta_R + \eta_L}\Big[  v_R - v_L \Big] .
\ea
\ese
It is seen that to leading order $v^I$ is an inverse impedance average, while $w^I$ is a impedance average. 
The second terms in~\eqref{eq:impedanceAverage} can usually be neglected as they are should proportional to the truncation error. 

The treatment of $u$ and $(p,q)$ at the interface is not as straightforward and we propose two possible approaches.
In each case we do not alter the polarization vectors and so these remain fixed. 

\bigskip 
\noindent Option 1: Given $D_L \eqdef \eps_L u_L + p_L $ and $D_R \eqdef \eps_R u_R + p_R$ choose the interface value $D^I$
to be the left or
right state based which side has a smaller $\eps_m$, 
\ba
D^I = \begin{cases}
          D_L, & \text{if $\eps_L \le \eps_R$}, \\
          D_R, & \text{if $\eps_R < \eps_L$} .
      \end{cases} 
\label{eq:dAveI}
\ea
Thus if  $\eps_R < \eps_L$ we take $D^I = D_R$ and
\bse
\bat
   &   u_L^I = \f{\eps_R}{\eps_L} u_R + \f{p_R-p_L}{\eps_L} , \quad&& u_R^I = u_R, \label{eq:option1a}
\eat
otherwise we take $D^I = D_L$ and
\bat
   &   u_L^I = u_L , \quad&&   u_R^I =\f{D^I-p_R}{\eps_R}  =
              \f{\eps_L}{\eps_R} u_L + \f{p_L-p_R}{\eps_R} , \label{eq:option1b} 
\eat
\ese
%--
%--\ba
%--D^I = D_R, \qquad\text{if $\eps_R < \eps_L$} \quad \implies u_L^I =\f{D^I-p_L}{\eps_L}  =
%--              \f{\eps_R}{\eps_L} u_R + \f{p_R-p_L}{\eps_L} , \label{eq:option1a}  , \\
%--D^I = D_L, \quad\text{if $\eps_L < \eps_R$} \quad \implies u_R^I =\f{D^I-p_R}{\eps_R}  =
%--              \f{\eps_L}{\eps_R} u_L + \f{p_L-p_R}{\eps_R} , \label{eq:option1b} 
%--\ea
%--\ba
%--   u^I = u_R \qquad\text{if $\eps_R < \eps_L$} \quad \implies u_L^I = \f{\eps_R}{\eps_L} u_R, \label{eq:option1a} \\
%--   u^I = u_L \qquad\text{if $\eps_L < \eps_RL$} \quad \implies u_R^I = \f{\eps_L}{\eps_R} u_L.\label{eq:option1b}
%--\ea
The intuition for this choice is based on a numerical stability argument. In assigning $u_L^I$ in~\eqref{eq:option1a}
any perturbations to $u_R \rightarrow u_R+\delta$ (e.g. round-off errors) will not be amplified when
multiplying by $\eps_R/\eps_L < 1$. 
% The intuition here is purely based on defining a stable assignment by using a ratio of $\eps$'s that is less than one.

\bigskip
\noindent Option 2: The variable $u$ (or $d\eqdef \eps u + p$) has a characteristic speed $\lambda=0$ and so the characteristics
do not immediately indicate how to project two values of $u$ on either side of the interface. In option 2 we consider what
happens if we add an upwind type dissipation to the $d$ equation, which then turns the equation into the heat equation, 
\ba
 %   \eps_m u_t = \p_x( \alpha_m  h_m \, \p_x u) ,  \\ 
&   \p_t d  = \p_x( \dc \, \p_x d_m) , \label{eq:heat}
\intertext{where the diffusion coefficient is taken to be of the form}
&    \dc = \begin{cases}
         \alpha_L c_L  h_L & \text{for $x<0$}, \\
         \alpha_R c_R  h_R & \text{for $x>0$}, \\
         \end{cases}  
\ea
where $\alpha_m$ is some non-dimensional parameter, $c_m = 1/\sqrt{\eps_m \mu_m}$ is chosen as
a velocity scale, and $h_m$ is a representative grid spacing.
% By dimensional arguments we take $\alpha_m \,\propto\, c_m$ where $c_m^2 = 1/(\eps_m \mu_m)$, so that
% \bas
%     \alpha_m = \beta_m \eps_m c_m = \beta_m \f{1}{\eta_m} 
% \eas
% for some non-dimensional $\beta_m$. 
First note that there is a similarity solution to the heat-equation given by {\red **CHECK ME***}
\bas
&     d(x,t) = A + B \, \erf\Big(  \f{x}{2\sqrt{\dc\, t}} \Big) , 
% &  \dc \eqdef \f{K}{\rho C_p} = \f{\beta_m \eps_m c_m}{\eps_m} = \beta_m c_m 
% &  \dc_m \eqdef \alpha_m c_m  h_m .
\eas
Now consider a Riemann problem for the heat equation for some left and right states,
\bas
d(x,0) = \begin{cases}
          D_L, & \text{if $x<0$}, \\
          D_R, & \text{if $x>0$} .
      \end{cases} 
\eas
The boundary conditions at infinity are
\bats
 &   d = D_R, \qquad&& x\rightarrow \infty, \\
 &   d = D_L, \qquad&& x\rightarrow -\infty,
\eats
which implies the solution takes the form (since $\erf(x)$ asymptotes to $\pm 1$ as $x\rightarrow \pm\infty$), 
\bats
&     d(x,t) = A_L + (A_L-D_L) \, \erf\Big(  \f{x}{2\sqrt{\dc_L\,  t}} \Big) , \qquad&& \text{for $x<0$},  \\
&     d(x,t) = A_R + (D_R-A_R) \, \erf\Big(  \f{x}{2\sqrt{\dc_R\,  t}} \Big) , \qquad&& \text{for $x>0$} .
\eats
The jump conditions at $x=0$ are 
\bas
  &   \jumpI{d} = 0  , \\
  &   [\dc \, \p_x d] = 0 ,
\eas
which gives (for $t>0$)
\bas
&     A_L =  A_R, \\
& \dc_L  (A_L-u_L) \f{1}{\sqrt{\dc_L}} = \dc_R  (u_R-A_R) \f{1}{\sqrt{\dc_R}}, 
% \implies ~~&   \sqrt{\dc_L}  (A_L-D_L)  =   \sqrt{\dc_R}  (D_R - A_R) , 
\eas
% This implies the interface value for $(\eps u)^I = \eps_L A_L = \eps_R A_R$ is
% a $\sqrt{\dc}=\sqrt{\beta c}$ weighted average of $\eps_R\,u_R$ and $\eps_L \,u_L$, 
which implies that the value at the interface, $x=0$, for $t>0$ is 
\bas
& D^I =  A_L = A_R
= \f{ \sqrt{\dc_R} \, D_R +  \sqrt{\dc_L}\, D_L }{ \sqrt{\dc_L}+\sqrt{\dc_R}},
% \qquad   \sqrt{\dc_m} = \sqrt{\alpha_m c_m  h_m}. 
\eas
Taking $ \alpha_L h_L = \alpha_R h_R$ as a typical case implies $D^I$ is an weighted average
$D_L$ and $D_R$, 
\ba
& D^I =  A_L = A_R
= \f{ \sqrt{c_R} \, D_R +  \sqrt{c_L}\, D_L }{ \sqrt{c_L}+\sqrt{c_R}},    \label{eq:dAveII}
\ea
with weights given by $\sqrt{c_m}$ . 
Equation~\eqref{eq:dAveII} shows that
the solution from the side with a larger value for
$c_m = 1/\sqrt{\eps_m \mu_m}$ (i.e. a smaller value for $\eps_m\mu_m$) is weighted more in the average.
This roughly corresponds to option 1 where the side with smaller $\eps_m$ was taken to define $D^I$.

%- 
%- Taking $\beta_L=\beta_R$ implies
%- that the solution from the side with a bigger $c_m$ (smaller $\eps_m\mu_m$) is weighted more in the average (and
%- thus roughly corresponds to option 1).
%- Solving for $u(0-,t)=A_L$ and $u(0+,t)=A_R$ gives **check me**
%- \bas
%-    u(0-,t)= \f{1}{\eps_L}  (\eps u)^I, \\
%-    u(0+,t)= \f{1}{\eps_R}  (\eps u)^I,
%- \eas
%- 
%- \bigskip
%- \textbf{Question:} Dispersive case (the question is what to do with the polarization $\Pv_m$?) 
%- %--\bas
%- %--&    \eps u_t = -\eps \p_t( \charv*u) = -  P_t  , \\
%- %--&    \eps v_t = - w_x-  \eps \p_t( \charv*v) = - w_x-  Q_t , \\
%- %--&    \mu w_t = - v_x 
%- %--\eas
%- %--One argument is that since the polarizaton is a time-convolution of the electric field, $\Pv=\eps \charv*\Ev$,
%- %--it will remain essentially constant over the infinitesimal time that the Riemann problem is solved. This implies
%- %--we do not project the polarization. 
%- 
%- Thus we project the tangential components of $\Ev$ using the inverse-impedance average given above.
%- Given $\Dv_m = \eps_m \Ev_m + \Pv_m$, we project the normal component of $D_m \eqdef \nv\cdot\Dv_m$ as
%- \bas
%-      D^I = \f{ \sqrt{\dc_R} (D_R) +  \sqrt{\dc_L}  (D_L)}{ \sqrt{\dc_L}+\sqrt{\dc_R}},
%- \eas
%- then set the projected values of $E$ on either side $\Ev_m^I$ by 
%- \bas
%-   & \eps_m \Ev_m^I + \Pv_m = D^I, \qquad m=L,R , \\
%- \implies \quad & \Ev_m^I = \f{D^I - \Pv_m}{\eps_m} 
%- \eas

\bigskip\noindent
\textbf{Summary.}
The tangential components of the electric field are projected using an
inverse impedance weighted average,
\bas
    \nv\times\Ev_m^{I} = \f{ \eta_L^{-1} \nv\times\Ev^L + \eta_R^{-1} \nv\times\Ev^R }{\eta_L^{-1} + \eta_R^{-1}} , \qquad \lr=L,R.
\eas
The normal component of the field is projected using
\bas
% &      D^I = \f{ \sqrt{c_R} \, D_R +  \sqrt{c_L}  \, D_L }{ \sqrt{c_L}+\sqrt{c_R}}, \\
&     \nv\cdot\Ev_\lr^I = \f{D^I - \nv\cdot\Pv_\lr}{\eps_\lr} , \qquad m=L,R .
\eas
where $D^I$ is defined by~\eqref{eq:dAveI} or~\eqref{eq:dAveII}.
