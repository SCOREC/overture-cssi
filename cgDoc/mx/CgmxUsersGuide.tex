%-----------------------------------------------------------------------
% User's Guide for CGMX -- Maxwell's Equation Solver
% 
%-----------------------------------------------------------------------
\documentclass{article}
\usepackage[bookmarks=true,colorlinks=true,linkcolor=blue]{hyperref}


% \input documentationPageSize.tex
\hbadness=10000 
\sloppy \hfuzz=30pt

% \voffset=-.25truein
% \hoffset=-1.25truein
% \setlength{\textwidth}{7in}      % page width
% \setlength{\textheight}{9.5in}    % page height

\usepackage{calc}
\usepackage[lmargin=.75in,rmargin=.75in,tmargin=.75in,bmargin=.75in]{geometry}

% \input homeHenshaw

\usepackage{amsmath}
\usepackage{amssymb}

\usepackage{verbatim}
\usepackage{moreverb}

% \usepackage{epsfig}    
% This next section will allow graphics files to be ps or pdf  -- from Jeff via Jeff
% \usepackage{ifpdf}
% \ifpdf
%     \usepackage[pdftex]{graphicx}
%     \usepackage{epstopdf}
%     \pdfcompresslevel=9
%     \pdfpagewidth=8.5 true in
%     \pdfpageheight=11 true in
%     \pdfhorigin=1 true in
%     \pdfvorigin=1.25 true in
% \else
%     \usepackage{graphicx}
% \fi

% \input{pstricks}\input{pst-node}
% \input{colours}

% define the clipFig commands:
\input ../common/trimFig.tex
\usepackage{xargs}% for optional args to \newcommandx
\input ../common/plotFigureMacros.tex

% --------------------------------------------
\usepackage[usenames]{color} % e.g. \color{red}
\newcommand{\red}{\color{red}}
\newcommand{\blue}{\color{blue}}
\newcommand{\green}{\color{green}}
\newcommand{\jwb}[2]{{\color{red}(old: #1) }{\color{green} #2}}

\usepackage{tikz}

\usepackage{listings}
\lstset{
basicstyle=\footnotesize\ttfamily,
columns=flexible,
breaklines=true,
commentstyle=\color{red},
keywordstyle=\color{black}\bfseries
}

\usepackage{makeidx} % index
\makeindex
\newcommand{\Index}[1]{#1\index{#1}}

\newcommand{\Gc}{{\mathcal G}}
% ---- we have lemmas and theorems in this paper ----
\newtheorem{assumption}{Assumption}
\newtheorem{definition}{Definition}

% \newcommand{\homeHenshaw}{/home/henshaw.0}

% \\newcommand{\primer}{/users/henshaw/res/primer}
% \\newcommand{\GF}{/users/\-henshaw/\-res/\-gf}
% \\newcommand{\gf}{/users/henshaw/res/gf}
% \\newcommand{\mapping}{/users/henshaw/res/mapping}

% \newcommand{\docFigures}{\homeHenshaw/OvertureFigures}
% \newcommand{\figures}{\homeHenshaw/res/OverBlown/docFigures}
% \newcommand{\obFigures}{\homeHenshaw/res/OverBlown/docFigures}  % note: local version for OverBlown
% \newcommand{\maxDoc}{\homeHenshaw/res/maxwell/doc}

% \\newcommand{\OVERTUREOVERTURE}{/users/\-henshaw/\-Overture/\-Overture}
% \\newcommand{\OvertureOverture}{/users/henshaw/Overture/Overture}

\newcommand{\Overture}{{\bf Overture\ }}
\newcommand{\OverBlown}{{\bf OverBlown\ }}
\newcommand{\overBlown}{{\bf overBlown\ }}


% *** See http://www.eng.cam.ac.uk/help/tpl/textprocessing/squeeze.html
% By default, LaTeX doesn't like to fill more than 0.7 of a text page with tables and graphics, nor does it like too many figures per page. This behaviour can be changed by placing lines like the following before \begin{document}

\renewcommand\floatpagefraction{.9}
\renewcommand\topfraction{.9}
\renewcommand\bottomfraction{.9}
\renewcommand\textfraction{.1}   
\setcounter{totalnumber}{50}
\setcounter{topnumber}{50}
\setcounter{bottomnumber}{50}


\begin{document}


% -----definitions-----
\input ../common/wdhDefinitions.tex

\def\ud     {{    U}}
\def\pd     {{    P}}

\newcommand{\mbar}{\bar{m}}
\newcommand{\Rbar}{\bar{R}}
\newcommand{\Ru}{R_u}         % universal gas constant
% \newcommand{\Iv}{{\bf I}}
% \newcommand{\qv}{{\bf q}}
\newcommand{\Div}{\grad\cdot}
\newcommand{\tauv}{\boldsymbol{\tau}}
\newcommand{\sumi}{\sum_{i=1}^n}
% \newcommand{\half}{{1\over2}}
\newcommand{\dt}{{\Delta t}}
\newcommand{\eps}{\epsilon}

\vglue 10\baselineskip
\begin{flushleft}
{\Large
Cgmx User Guide: An Overture Solver for Maxwell's Equations on Composite Grids \\
}
\vspace{2\baselineskip}
William D. Henshaw  \\
Department of Mathematical Sciences, \\
Rensselaer Polytechnic Institute, \\
Troy, NY, USA, 12180.
% Centre for Applied Scientific Computing  \\
% Lawrence Livermore National Laboratory      \\
% Livermore, CA, 94551.  \\
% \vspace{\baselineskip}
% LLNL-SM-523971 \\
% henshaw@llnl.gov \\
% http://www.llnl.gov/casc/people/henshaw \\
% http://www.llnl.gov/casc/Overture\\
\vspace{\baselineskip}
\today\\
\vspace{\baselineskip}
% UCRL-MA-134288

\vspace{4\baselineskip}

\noindent{\bf\large Abstract:}

Cgmx is a program that can be used to solve the time-dependent Maxwell's equations
of electromagnetics on composite overlapping grids in two and three space
dimensions. This document explains how to run Cgmx. Numerous examples are
given. The boundary conditions, initial conditions and forcing functions 
are described.
Cgmx solves Maxwell's equations in second-order form. Second-order
accurate and fourth-order accurate approximations are available. Cgmx can
accurately handle material interfaces.  Cgmx also implements a version of the
Yee scheme for the first-order system form of Maxwell's equations. The Yee
scheme can be run on a single Cartesian grid with material boundaries treated
with a stair-step approximation.

\end{flushleft}

\clearpage
\tableofcontents
% \listoffigures


\vfill\eject

\section{Introduction}

   Cgmx is a program that can be used to solve Maxwell's equations of electromagnetics 
on composite overlapping grids~\cite{max2006b}. It is built upon
the \Overture framework~\cite{Brown97},\cite{Henshaw96a},\cite{iscope97}.

More information about
{\bf Overture} can be found on the \Overture home page, {\tt overtureFramework.org}.


\noindent{\bf Cgmx features:}

\begin{itemize}
  \item solve the time-domain Maxwell equations on overlapping grids to second- and fourth-order accuracy in two and three space dimensions. 
  \item solve material interface problems to second- and fourth-order accuracy in two and three space dimensions. 
  \item solve the time-domain Maxwell equations on a Cartesian grid with variable $\mu$ and $\epsilon$ using the Yee scheme.
\end{itemize}


\noindent{\bf Installation:} To install cgmx you should follow the instructions for installing Overture and cg
from the Overture web page. Note that cgmx is NOT built by default when building the cg solvers so that
after installing cg you must go into the {\tt cg/mx} directory and type `make'. 
If you only wish to build cgmx and not any of the other cg solvers, then
after building Overture and unpacking cg, go to the {\tt cg/mx} and type make. 

\noindent The cgmx solver is found in the {\tt mx} directory in the {\bf cg} distribution and has
sub-directories
\begin{description}
 \item[{\tt bin}] : contains the executable, cgmx. You may want to put this directory in your path.
 \item[{\tt check}] : contains regression tests.
 \item[{\tt cmd}] : sample command files for running cgmx, see section (\ref{sec:demo}).
 \item[{\tt doc}] : documentation.
 \item[{\tt lib}] : contains the cgmx library, {\tt libCgmx.a}.
 \item[{\tt src}] : source files 
\end{description}


\subsection{Basic steps}\index{basic steps}
Here are the basic steps to solve a problem with cgmx.
\begin{enumerate}
  \item Generate an overlapping grid with ogen. 
  \item Run cgmx (found in the {\tt bin/cgmx} directory).
  \item Assign the boundary conditions and initial conditions.
  \item Choose the parameters for the PDE (such as material properties such as $\mu$, $\epsilon$))
  \item Choose run time parameters, time to integrate to, time stepping method etc.
  \item Compute the solution (optionally plotting the results as the code runs).
  \item When the code is finished you can look at the results (provided you saved a
     `show file') using {\tt plotStuff}.
\end{enumerate}
The commands that you enter to run cgmx can be saved in a \Index{command file} (by default
they are saved in the file `cgmx.cmd'). This command file can be used to re-run
the same problem by typing `cgmx file.cmd'. The command file can be edited to change parameters.

To get started you can run one of the demo's that come with cgmx, these are 
explained in section~(\ref{sec:demo}).
% 
For more information on the algorithms and approximations used in Cgmx see
\begin{enumerate}
  \item The {\sl Cgmx Reference Manual}~\cite{CgmxReferenceManual}.
  \item {\sl A High-Order Accurate Parallel Solver for {Maxwell}'s Equations on Overlapping Grids}
        \cite{max2006b}.
\end{enumerate}

  
% \begin{figure}[hbt]
% \begin{center}
%   \epsfig{file=\obFigures/OverBlownScreen.ps,width=.95\linewidth}  \\
% \caption{Snapshot of cgmx showing the run time dialog menu. 
%     The figure shows two falling bodies in an incompressible flow,
%     computed with the command file twoDrop.cmd.}
%   \end{center} 
%   \label{fig:screenDrops}
% \end{figure}

% ===============================================================================
\clearpage
\input tex/sampleCommandFiles



% =======================================================================================================
\clearpage
\section{Boundary Conditions} \label{sec:bc}

The general format for setting a boundary condition in a command file is
\begin{verbatim}
  bc: name=bcName
\end{verbatim}
where "name" specifies the name of a grid face, or grid and "bcName" is the name of the boundary
condition (given below), 
\begin{verbatim}
  name = [all][gridName][gridName([0|1],[0|1|2])]
  bcName=[dirichlet][perfectElectricalConductor][planeWaveBoundaryCondition][abcEM2]...
\end{verbatim}
Here are some examples of setting boundary conditions in a command file,
\begin{verbatim}
  bc: all=perfectElectricalConductor
  bc: backGround=abcEM2
  bc: backGround(0,0)=planeWaveBoundaryCondition
  bc: square(1,0)=planeWaveBoundaryCondition
\end{verbatim}
\begin{enumerate}
  \item Note that {\em backGround} and {\em square} are the names of the grids (specified when the grid was constructed with Ogen).
  \item Note that {\em backGround(0,0)} specifies the {\em left} face of the  grid {\em backGround} using the standard Overture conventions for
   specifying the face of a grid as {\em gridName(side,axis)}.
  \item Note that the special name {\em all} means apply the boundary condition to all faces of all grids.
  \item The last boundary condition given to a face is the one that will be used. 
\end{enumerate}

\noindent The following boundary conditions are (more or less) available with cgmx,
\begin{description}
  \item[periodic]: chosen automatically to match the grid created with Ogen.
  \item[dirichlet]: a non-physical boundary condition where all components of the field are set to some known function.
                    For example, the known function could be an exact solution or a twilight-zone function.
  \item[perfectElectricalConductor]: PEC boundary condition, see Section~\ref{sec:perfectElectricalConductor}.
  \item[perfectMagneticConductor]: PMC boundary condition (NOT implemented yet).
  \item[planeWaveBoundaryCondition]: solution on the boundary is set equal to a plane wave solution, see Section~\ref{sec:planeWaveBoundaryCondition}. 
  \item[symmetry]: apply a symmetry boundary condition. 
  \item[interfaceBoundaryCondition]: for the interface between two regions with different properties
  \item[abcEM2]: absorbing BC, Engquist-Majda order 2, see Section~\ref{sec:EngquistMajdaABC}.
  \item[abcPML]: perfectly matched layer far field condition, see Section~\ref{sec:PML}.
%  \item[abc3]: future absorbing BC
%   \item[abc4]: future absorbing BC
%   \item[abc5]: future absorbing BC
  \item[rbcNonLocal]: radiation BC, non-local
  \item[rbcLocal]: radiation BC, local
\end{description}


% --------------------------------------------------------------------------------
\subsection{Perfect electrical conductor boundary condition}\label{sec:perfectElectricalConductor}

The perfect electrical conductor boundary condition sets the tangential components of the
electric field to zero
\begin{align}
   \tauv_m\cdot\Ev &= 0, \qquad \text{for $\xv \in \partial\Omega_{\rm pec}$}. 
\end{align}
Here $\tauv_m$, $m=1,2$, denote the tangent vectors to the boundary surface, $\partial\Omega_{\rm pec}$.

% --------------------------------------------------------------------------------
\subsection{Plane wave boundary condition}\label{sec:planeWaveBoundaryCondition}

The plane wave boundary condition set the electric (and magnetic) fields on the boundary to the plane wave 
solution~\eqref{eq:planeWaveE}-\eqref{eq:planeWaveH} defined in Section~\ref{sec:planeWaveIC},
\begin{align}
   \Ev(\xv,t) &= \Ev_{\rm pw}(\xv,t), \qquad \text{for $\xv \in \partial\Omega_{\rm pw}$}, \\
   \Hv(\xv,t) &= \Hv_{\rm pw}(\xv,t), \qquad \text{for $\xv \in \partial\Omega_{\rm pw}$}. 
\end{align}

% --------------------------------------------------------------------------------
\subsection{Engquist-Majda absorbing boundary conditions}\label{sec:EngquistMajdaABC}


The boundary condition {\tt abcEM2} uses the Engquist-Majda absorbing boundary 
condition (defined here of a boundary $x={\rm constant}$), 
\begin{align}
   \partial_t\partial_x u = \alpha \partial_x^2 u + \beta (\partial_y^2+\partial_z^2) u 
\end{align}
With $\alpha=c$ and $\beta=\half c$, this gives a {\em second-order accurate} approximation to 
a pseudo-differential operator that absorbs outgoing traveling waves. 
Here $u$ is any field which satisfies the second-order wave equation.

% --------------------------------------------------------------------------------
\subsection{Perfectly matched layer boundary condition}\label{sec:PML}

The boundary condition {\tt abcPML} imposes a perfectly matched layer boundary condition.
With this boundary condition, auxiliary equations are solved over a layer (of some number of
specified grid points) next to the boundary. The PML equations we solve
are those suggested in Hagstrom~\cite{Hagstrom1999} and given by (defined here for a boundary $x={\rm constant}$), (*check me*)
\begin{align}
  u_{tt} &= c^2 \Big( \Delta u - \partial_x v - w \Big), \\
   v_t &= \sigma( -v + \partial_x u ) , \\
   w_t &= \sigma ( -w  - \partial_x v + \partial_x^2 u ). 
\end{align}
Here $u$ is any field which satisfies the second-order wave equation and $v$ and $w$ are auxiliary variables
that only live in the layer domain. 
The PML damping function $\sigma_1(\xi)$ is given by 
\begin{align}
  \sigma(\xi) = a \xi^p
\end{align}
where $a$ is the strength, $p$ is the power and where $\xi$ varies from $0$ to $1$ through the layer.




% ------------------------------------------------------------------------------
% \subsection{Far field boundary conditions} \label{sec:farFieldBC}


% =======================================================================================================
\clearpage
\input initialConditions

% =======================================================================================================
\clearpage
\input forcingFunctions

% =======================================================================================================
\clearpage
\input boundaryForcingFunctions

% =======================================================================================================
% \clearpage
% \section{Variable material properties} \label{sec:varMat}
% 
% Finish me...




% =======================================================================================================
\clearpage
\section{Options} \label{sec:option}

% ---------------------------------------------------------------------------
\subsection{Options affecting the scheme}

\begin{description}
  \item [\qquad cfl] $value$ : set the CFL number. The schemes are usually stable to CFL=1. The default is CFL=.9.
  \item [\qquad order of dissipation] $[2|4|6]$ : set the order of the dissipation.
  \item [\qquad dissipation] $value$ : set the coefficient of the dissipation (e.g. 1). Dissipation is usually needed on overlapping grids. 
  \item [\qquad coefficients] $\eps~\mu~gridName$: specify the values of $\eps$ and $\mu$ on a grid.
  \item [\qquad adjustFarFieldBoundariesForIncidentField] $[0|1]~[gridName|all]$ : subtract off the plane-wave solution before applying the
         far-field boundary condition.
\end{description}


% ---------------------------------------------------------------------------
\subsection{Run time options} \label{sec:runTimeDialog}

\begin{description}
  \item [\qquad tFinal] $value$ : solve the equations to this time.
  \item [\qquad tPlot] $value$ : time increment to save results and/or plot the solution.
  \item [\qquad debug] $value$ : an integer bit flag that turns on debugging information. For example, set debug=1 for some info, debug=3 (=1+2) for some
    more, debug=7 (=1=2+4)for even more. 
\end{description}

% ---------------------------------------------------------------------------
\subsection{Plotting options}

\begin{description}
  \item [\qquad error norm] $[0|1|2]$ : compute errors in the max-norm, $L_1$-norm or $L_2$-norm. (*check me*)
  \item [\qquad plot scattered field] $[0|1]$ : when computing the scattered field directly use this option to plot the scattered field.
  \item [\qquad plot total field] $[0|1]$ : when computing the scattered field directly use this option to plot the total field (i.e. add in the
            plane wave solution before plotting).
  \item [\qquad plot errors] $[0|1]$ :
  \item [\qquad check errors] $[0|1]$ :
  \item [\qquad plot intensity] $[0|1]$ : plot the intensity (really only make senses for fields that are time-harmonic). 
  \item [\qquad plot harmonic E field] $[0|1]$ : plot the components of the complex harmonic fields (for problems that are time-harmonic).
\end{description}

% ---------------------------------------------------------------------------
\subsection{Output options}

\begin{description}
  \item [\qquad specify probes] : specify a list of probe locations as $x$, $y$, $z$ values, one per line, finishing the
         list with 'done'. The solution values at these locations (actually the closest grid point to each location, with these
         values written to the screen) are
    written to a text file whose default name is "probeFile.dat". 
    In the following example we specify two probe locations, $(.2,.3,.1)$ and $(.4,.6,.3)$, 
\begin{verbatim}
specify probes
  .2 .3 .1.
  .4 .6 .3
done
\end{verbatim}
    Each line of the probe file contains the time followed
    by the three components of $\Ev$ (or in two-dimensions $E_x$, $E_y$ and $H_z$) for each probe location. For example,
    with two probes specified the file in three dimensions would contain data of the form 
\begin{verbatim}
   t1 Ex11 Ey11 Ez11  Ex12 Ey12 Ez12 
   t2 Ex21 Ey21 Ez21  Ex22 Ey22 Ez22 
   ...
\end{verbatim}
  \item [\qquad probe file:] $name$ : specify the name of the probe file.
  \item [\qquad probe frequency] $value$ : specify the frequency at which values are saved in the probe file. For example,
      if the probe frequency is set to 2 then the solution will be saved every 2nd time step to the probe file. 
\end{description}

% =======================================================================================================
\clearpage
\section{Maxwell's Equations} \label{sec:equations}


The time dependent Maxwell's equations for linear, isotropic and non-dispersive materials are
\begin{align}
  \partial_t \Ev &=  {1\over \eps} \grad\times\Hv - {1\over \eps}\Jv , \label{eq:FOS-Et}  \\
  \partial_t \Hv &= - {1\over \mu} \grad\times\Ev ,  \label{eq:FOS-Ht} \\
  \grad\cdot(\eps\Ev) &=\rho , ~~ \grad\cdot(\mu\Hv) = 0 , \label{eq:FOS-div}
\end{align}
Here $\Ev=\Ev(\xv,t)$ is the electric field, 
$\Hv=\Hv(\xv,t)$ is the magnetic field, $\rho=\rho(\xv,t)$ is the electric charge density,
$\Jv=\Jv(\xv,t)$ is the electric current density,
$\eps=\eps(\xv)$ is the electric permittivity, and $\mu=\mu(\xv)$ is the magnetic permeability.
This first-order system for Maxwell's equations can also be written in a
second-order form. By taking the time derivatives of~(\ref{eq:FOS-Ht}) and
(\ref{eq:FOS-Et}) and using (\ref{eq:FOS-div}) it follows that 
\begin{align}
 \eps\mu~\partial_t^2 \Ev &= \Delta \Ev + \grad\Big( \grad \ln\eps~\cdot\Ev \Big)
        +\grad\ln\mu\times\Big(\grad\times\Ev\Big) 
            -\grad(\frac{1}{\epsilon}\rho)- \mu \partial_t\Jv , \label{eq:waveEGen} \\
 \eps\mu~\partial_t^2 \Hv &= \Delta \Hv + \grad\Big( \grad \ln\mu~\cdot\Hv \Big)
                               +\grad\ln\eps\times\Big(\grad\times\Hv\Big) 
                     + \eps\grad\times(\frac{1}{\epsilon}\Jv ) \label{eq:waveHGen}.
\end{align}
It is evident that the equations for the electric and magnetic field are decoupled with each 
satisfying a vector wave equation with lower order terms.
In the case of constant $\mu$ and $\eps$ and no charges, $\rho=\Jv=0$, 
the equations simplify to the classical second-order wave equations,
\begin{align}
  \partial_t^2 \Ev = c^2~ \Delta \Ev , \qquad
  \partial_t^2 \Hv = c^2~ \Delta \Hv \label{eq:waveE}
\end{align}
where $c^2=1/(\eps\mu)$.
There are some advantages to solving the second-order form of the equations
rather than the first-order system. One advantage is that in some cases it is
only necessary to solve for one of the variables, say $\Ev$. 
If the other variable, $\Hv$ is required, it can be
determined by
integrating equation~\eqref{eq:FOS-Ht} as an ordinary differential equation
with known $\Ev$. Alternatively, as a post-processing step $\Hv$ can be computed from an
elliptic boundary value problem formed by taking the curl of equation~\eqref{eq:FOS-Et}.
Another advantage of the second-order form, which simplifies the implementation on
an overlapping grid, is that there is no need to use a staggered grid formulation. 
Many schemes approximating the first order system~(\ref{eq:FOS-Ht}-\ref{eq:FOS-div}) rely on a
staggered arrangement of the components of $\Ev$ and $\Hv$ such as the
popular Yee scheme~\cite{Yee66} for Cartesian grids. 


% ===============================================================================================
\input acknowledgments


\vfill\eject
% \bibliography{\homeHenshaw/papers/common/henshaw,\homeHenshaw/papers/common/henshawPapers}
\bibliography{../common/henshaw,../common/henshawPapers}
\bibliographystyle{siam}

\printindex

\end{document}

% ***************************************************************************************************




