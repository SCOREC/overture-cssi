%=======================================================================================================
% MatrixMotions : describe the features of the MatrixMotion class that can be used to define
%  rigid body motions
%=======================================================================================================

% -- article: 
% \documentclass[10pt]{article}
% 
% \hbadness=10000 
% \sloppy \hfuzz=30pt
% \usepackage{calc}
% 
% % set the page width and height for the paper (The covers will have their own size)
% \setlength{\textwidth}{7in}  
% \setlength{\textheight}{9.5in} 
% % here we automatically compute the offsets in order to centre the page
% \setlength{\oddsidemargin}{(\paperwidth-\textwidth)/2 - 1.in}
% % \setlength{\topmargin}{(\paperheight-\textheight -\headheight-\headsep-\footskip)/2 - 1in + .5in }
% \setlength{\topmargin}{(\paperheight-\textheight -\headheight-\headsep-\footskip)/2 - 1.25in + .8in }
% 
% \input homeHenshaw
% 
% \usepackage{color}
% 
% \input wdhDefinitions.tex
% 
% \input{pstricks}\input{pst-node}
% \input{colours}
% 
% \newcommand{\bogus}[1]{}  % begin a section that will not be printed
% 
% 
% \usepackage{amsmath}
% \usepackage{amssymb}
% 
% % \usepackage{verbatim}
% % \usepackage{moreverb}
% % \usepackage{graphics}    
% \usepackage{epsfig}    
% % \usepackage{calc}
% % \usepackage{ifthen}
% % \usepackage{float}
% % \usepackage{fancybox}
% 
% % define the clipFig commands:
% \input clipFig.tex
% 
% % \newcommand{\grad}{\nabla}
% % \newcommand{\half}{\frac{1}{2}}}
% 
% \newcommand{\eps}{\epsilon}
% 
% \newcommand{\bc}[1]{\mbox{\bf#1}}   % bold name
% \newcommand{\cc}[1]{\mbox{  : #1}}  % comment
% 
% \newcommand{\Overture}{{Overture}}
% 
% \newcommand{\Largebf}{\sffamily\bfseries\Large}
% \newcommand{\largebf}{\sffamily\bfseries\large}
% \newcommand{\largess}{\sffamily\large}
% \newcommand{\Largess}{\sffamily\Large}
% \newcommand{\bfss}{\sffamily\bfseries}
% \newcommand{\smallss}{\sffamily\small}
% \newcommand{\normalss}{\sffamily}
% \newcommand{\scriptsizess}{\sffamily\scriptsize}
% 
% 
% % -----definitions-----
% 
% \psset{xunit=1.cm,yunit=1.cm,runit=1.cm}
% 
% 
% % *** See http://www.eng.cam.ac.uk/help/tpl/textprocessing/squeeze.html
% % By default, LaTeX doesn't like to fill more than 0.7 of a text page with tables and graphics, nor does it like too many figures per page. This behaviour can be changed by placing lines like the following before \begin{document}
% 
% \renewcommand\floatpagefraction{.9}
% \renewcommand\topfraction{.9}
% \renewcommand\bottomfraction{.9}
% \renewcommand\textfraction{.1}   
% \setcounter{totalnumber}{50}
% \setcounter{topnumber}{50}
% \setcounter{bottomnumber}{50}
% 
% 
% \begin{document}
% 
% % -----------------
% \title{MatrixMotion and TimeFunction: Classes for Defining Rigid Motions of Bodies for Overture and Cg}
% \author{William D. Henshaw\\
% \small Centre for Applied Scientific Computing,\\[-0.8ex]
% \small Lawrence Livermore National Laboratory, \\[-0.8ex]
% \small Livermore, CA 94551.\\
% \small \texttt{henshaw1@llnl.gov}
% }
% 
% \maketitle
% 
% \begin{abstract}
%    The {\tt MatrixMotion} C++ class can be used to define the motions of rigid bodies for use with the
% Cg partial differentoal equation solvers such as Cgins for incompressible flow and Cgcssi for compressible flow. 
% The MatrixMotion class allows one to rotate an object around an arbitray line in space 
% or to translate along a line. These motions can be composed together and thus one could
% have a rotation followed by a translation followed by another rotation. The MatrixMotion
% class uses the {\tt TimeFunction} C++ class to define how the rotation angle depends on time or how
% the translation distance depends on time. 
% 
% \end{abstract}
% 
% \tableofcontents
% 

\section{MatrixMotion and TimeFunction: Classes for Defining Rigid Motions of Bodies}\label{sec:matrixMotion}


   The {\tt MatrixMotion} C++ class can be used to define the motions of rigid bodies for use with the
CG partial differential equation solvers such as Cgins for incompressible flow and Cgcssi for compressible flow. 
The MatrixMotion class allows one to rotate an object around an arbitray line in space 
or to translate along a line. These motions can be composed together and thus one could
have a rotation followed by a translation followed by another rotation. The MatrixMotion
class uses the {\tt TimeFunction} C++ class to define how the rotation angle depends on time or how
the translation distance depends on time. 

The general motion of a solid body is defined by the matrix transformation
\begin{align}
  \xv(t) = R(t)\, \xv(0) + \gv(t),   \label{eq:matrixTransform}
\end{align}
where $\xv (t)\in \Real^3$ defines a point on the body at time $t$, 
$R(t)\in \Real^{3\times3}$ is a $3\times3$ {\em rotation} matrix and
$\gv\in\Real^3$ is a translation. Note that the matrix transformation~\eqref{eq:matrixTransform} is
implemented in the {\tt MatrixTransform} Mapping class~\cite{MAPPINGS} which can be used to rotate
and translate another Mapping. Also note that $R(t)$ can be any invertible matrix and thus does not
necessarily need to be a rotation although we will often refer to it as the {\em rotation matrix}.


\subsection{Elementary rigid motions}

\newcommand{\ct}{\cos(\theta)}
\newcommand{\st}{\sin(\theta)}
\subsubsection{Rotation around a line}
An elementary rigid motion is the rotation about an arbitrary line in space. Define a line by a point
on the line, $\xv_0$, and a tangent $\vv=[v_0, v_1, v_2]^T$ to the line,
\begin{align}
  \yv(s) = \xv_0 + \vv\, s  . \qquad\text{(line)}
\end{align}
The equation to rotate a given point $\xv(0)$ around this line by an angle $\theta$ to the
new point $\xv(\theta)$ is (for a derivation see the documentation of the {\tt RevolutionMapping} in~\cite{MAPPINGS}) 
\begin{align}
  \xv (\theta) &= R_l(\theta) (\xv(0)-\xv_0) + \xv_0,  \label{eq:lineRotationMotion} \\
   R_l(\theta) &= \vv \vv^T + \cos(\theta) ( I -\vv \vv^T ) + \sin(\theta) ( \vv \times )(  I -\vv \vv^T),\\
      &= \begin{bmatrix}
             v_0 v_0 (1-\ct)+\ct      &  v_0 v_1 (1-\ct)-\st v_2  & v_0 v_2 (1-\ct)+\st v_1  \\
             v_0 v_1 (1-\ct)+\st v_2  &  v_1 v_1 (1-\ct)+\ct      & v_1 v_2 (1-\ct)-\st v_0  \\
             v_0 v_2 (1-\ct)-\st v_1  &  v_2 v_1 (1-\ct)+\st v_0  & v_2 v_2 (1-\ct)+\ct   .  
         \end{bmatrix}
\end{align}
We can write~\eqref{eq:lineRotationMotion} in the form~\eqref{eq:matrixTransform} if we set
$R=R_l$ and $\gv = (I-R_l)\xv_0$.


\subsubsection{Translation along a line}
Another elementary rigid motion is the translation motion,
\begin{align}
  \xv(t) &= \av_0 + \vv\, f(t), \qquad\text{(translation along a line)} \label{eq:translationMotion}
\end{align}
where $f(t)$ is some {\em time function} that defines the position of the point along the line. The
translation motion~\eqref{eq:translationMotion} can also be put in the form~\eqref{eq:matrixTransform}
by setting $R=I$ and $\gv=\av_0 + \vv\, f(t)$. 

\subsection{Composition of motions}

One can compose multiple elementary rigid motions to form more complex motions.
If we apply the motion $\xv = R_1 \xv_0 + \gv_1$ followed by the motion $\xv = R_2 \xv_0 + \gv_2$
then we get
\begin{align}
  \xv(t) &= R_2 \Big( R_1 \xv_0 + \gv_1 \Big) + \gv_2, \\
         &= R_2 R_1 \xv_0 + R_2\gv_1 + \gv_2 ,
\end{align}
and thus the composed motion is defined by 
\begin{align}
  \xv(t) &= R \xv_0 + \gv, \\
     R    &= R_2 R_1 , \\
     \gv &= R_2\gv_1 + \gv_2. 
\end{align}

\subsection{Grid velocity and acceleration}

When the matrix motions are used with a PDE solver, we may need to know the velocity and
acceleration of a grid point. These are just the first and second time derivatives of the
motion, 
\begin{align}
  \dot{\xv}(t) &= \dot{R} \xv_0 + \dot{\gv}, \qquad\text{(velocity)} \\
  \ddot{\xv}(t) &= \ddot{R} \xv_0 + \ddot{\gv}, \qquad\text{(acceleration)}
\end{align}
where ``dot'' denotes a derivative with respect to time. 
These expressions can also be rewritten using $\xv(0) = R^{-1}(\xv(t)-\gv(t))$
to give the velocity and acceleration in terms of the current position
\begin{align}
  \dot{\xv}(t) &= \dot{R}R^{-1}(\xv(t)-\gv(t)) + \dot{\gv}, \\
  \ddot{\xv}(t) &= \ddot{R}R^{-1}(\xv(t)-\gv(t))+ \ddot{\gv}. 
\end{align}
These last expressions are useful if we do not want to save the original grid positions. 

\subsection{MatrixMotion class} \label{sec:MatrixMotion}

The {\tt MatrixMotion} C++ class can be used to define elementary rigid motions. A MatrixMotion
can be composed with another MatrixMotion and thus more general rigid motions can be defined.
The {\tt MatrixMotion} holds a {\tt TimeFunction} object (see Section~\ref{sec:TimeFunction}) that
defines the time function for the motion.



\subsection{TimeFunction class} \label{sec:TimeFunction}

The {\tt TimeFunction} C++ class is used to define functions of time that can be used with the
{\tt MatrixMotion} class to define, for example, the rotation angle $\theta(t)$ as a function of time.
The simplest time function is the {\tt linear function}
\begin{align}
  f(t) &= a_0 + a_1 t.
\end{align}
The {\tt sinusoidal function} is defined as
\begin{align}
  f(t) &= b_0 \sin( 2 \pi f_0( t-t_0)) .
\end{align}




\subsection{The `motion' program for building and testing motions} \label{sec:motion}

The test program {\tt motion} (type `make motion' in {\tt cg/user} to build 
{\tt cg/user/bin/motion} from {\tt cg/user/src/motion.C}) 
can be used to build and test rigid motions. One can
\begin{itemize}
  \item Build multiple bodies (e.g. ellipse, cylinder).
  \item Define a MatrixMotion and TimeFunction for each body.
  \item Compose multiple MatrixMotion's to build more complicated motions.
  \item Plot the motions of the bodies over time.
  \item Check the time derivatives of the motions by finite differences. 
  \item Evaluate the grid velocity and grid acceleration (as needed by {\tt Cgins} or {\tt Cgcssi})
        and check the accuracy of the velocity and acceleration by finite differencing the grid positions
        in time. 
\end{itemize}



