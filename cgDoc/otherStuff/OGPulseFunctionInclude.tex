\subsubsection{Constructor}
 
\newlength{\OGPulseFunctionIncludeArgIndent}
\begin{flushleft} \textbf{%
\settowidth{\OGPulseFunctionIncludeArgIndent}{OGPulseFunction(}% 
OGPulseFunction(int numberOfDimensions\_  = 2,\\ 
\hspace{\OGPulseFunctionIncludeArgIndent}int numberOfComponents\_  =1,\\ 
\hspace{\OGPulseFunctionIncludeArgIndent}real a0\_  =1., \\ 
\hspace{\OGPulseFunctionIncludeArgIndent}real a1\_  =5.,\\ 
\hspace{\OGPulseFunctionIncludeArgIndent}real c0\_  =0.,\\ 
\hspace{\OGPulseFunctionIncludeArgIndent}real c1\_  =0.,\\ 
\hspace{\OGPulseFunctionIncludeArgIndent}real c2\_  =0.,\\ 
\hspace{\OGPulseFunctionIncludeArgIndent}real v0\_  =1.,\\ 
\hspace{\OGPulseFunctionIncludeArgIndent}real v1\_  =1.,\\ 
\hspace{\OGPulseFunctionIncludeArgIndent}real v2\_  =1.,\\ 
\hspace{\OGPulseFunctionIncludeArgIndent}real p\_   =1.)
}\end{flushleft}
\begin{description}
\item[{\bf Description:}]  
    
    Define a pulse.

  \begin{align*}
    U &=  a_0 \exp( - a_1 | \xv-\bv(t) |^{2p} )  \\
    \bv(t) &= \cv_0 + \vv t
  \end{align*}

\item[{\bf numberOfDimensions\_ (input):}]  number of space dimensions, 1,2, or 3.
\item[{\bf numberOfComponents\_ (input):}]  maximum number of components required.
\item[{\bf a0\_,a1\_,...p\_ (input):}]  pulse parameters
 

\end{description}
\subsubsection{setRadius}
 
\begin{flushleft} \textbf{%
int  \\ 
\settowidth{\OGPulseFunctionIncludeArgIndent}{setRadius(}%
setRadius( real radius )
}\end{flushleft}
\begin{description}
\item[{\bf Description:}] 
  Set the approximte radius of the pulse. This will set the parameter $a_1$
   according to the formula $\mbox{radius} = 1/\sqrt{a_1}$.
  
\item[{\bf radius (input):}]  approximate radius.
 
\item[{\bf Author:}]  WDH
\end{description}
\subsubsection{setRadius}
 
\begin{flushleft} \textbf{%
int  \\ 
\settowidth{\OGPulseFunctionIncludeArgIndent}{setShape(}%
setShape( real p\_ )
}\end{flushleft}
\begin{description}
\item[{\bf Description:}] 
  Set the {\it shape} parameter p. $p=1$ gives a Gaussian pulse, choosing a larger value of $p$
  will cause the pulse to flatten on the top and approach a top-hat function as $p$ tends to infinity.
  
\item[{\bf p\_ (input):}]  shape parameter, $p>{1\over2}$.
 
\item[{\bf Author:}]  WDH
\end{description}
\subsubsection{setCentre}
 
\begin{flushleft} \textbf{%
int  \\ 
\settowidth{\OGPulseFunctionIncludeArgIndent}{setCentre(}%
setCentre( real c0\_   =0.,\\ 
\hspace{\OGPulseFunctionIncludeArgIndent}real c1\_   =0.,\\ 
\hspace{\OGPulseFunctionIncludeArgIndent}real c2\_   =0.)
}\end{flushleft}
\begin{description}
\item[{\bf Description:}] 
  Set the pulse centre.
  
\item[{\bf c0\_,c1\_,c2\_ (input):}]  centre.
 
\item[{\bf Author:}]  WDH
\end{description}
\subsubsection{setVelocity}
 
\begin{flushleft} \textbf{%
int  \\ 
\settowidth{\OGPulseFunctionIncludeArgIndent}{setVelocity(}%
setVelocity( real v0\_   =1.,\\ 
\hspace{\OGPulseFunctionIncludeArgIndent}real v1\_   =1.,\\ 
\hspace{\OGPulseFunctionIncludeArgIndent}real v2\_   =1.)
}\end{flushleft}
\begin{description}
\item[{\bf Description:}] 
  Set the pulse velocity.
  
\item[{\bf v0\_,v1\_,v2\_ (input):}]  velocity.
 
\item[{\bf Author:}]  WDH
\end{description}
\subsubsection{setCoefficients}
 
\begin{flushleft} \textbf{%
void  \\ 
\settowidth{\OGPulseFunctionIncludeArgIndent}{setParameters(}%
setParameters( int numberOfDimensions\_  = 2,\\ 
\hspace{\OGPulseFunctionIncludeArgIndent}int numberOfComponents\_  =1,\\ 
\hspace{\OGPulseFunctionIncludeArgIndent}real a0\_  =1., \\ 
\hspace{\OGPulseFunctionIncludeArgIndent}real a1\_  =5.,\\ 
\hspace{\OGPulseFunctionIncludeArgIndent}real c0\_  =0.,\\ 
\hspace{\OGPulseFunctionIncludeArgIndent}real c1\_  =0.,\\ 
\hspace{\OGPulseFunctionIncludeArgIndent}real c2\_  =0.,\\ 
\hspace{\OGPulseFunctionIncludeArgIndent}real v0\_  =1.,\\ 
\hspace{\OGPulseFunctionIncludeArgIndent}real v1\_  =1.,\\ 
\hspace{\OGPulseFunctionIncludeArgIndent}real v2\_  =1.,\\ 
\hspace{\OGPulseFunctionIncludeArgIndent}real p\_   =1.)
}\end{flushleft}
\begin{description}
\item[{\bf Description:}]  Use this member function to set parameters.

    Define a pulse.
  \begin{align*}
    U &=  a_0 \exp( - a_1 | \xv-\cv(t) |^p )  \qquad p>{1\over2}\\
    \cv(t) &= \cv_0 + \vv t
  \end{align*}

\item[{\bf numberOfDimensions\_ (input):}]  number of space dimensions, 1,2, or 3.
\item[{\bf numberOfComponents\_ (input):}]  maximum number of components required.
\item[{\bf a0\_,a1\_,...p\_ (input):}]  pulse parameters. 
 
\item[{\bf Author:}]  WDH
\end{description}
