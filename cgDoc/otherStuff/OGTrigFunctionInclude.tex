\subsubsection{Constructors}
 
\newlength{\OGTrigFunctionIncludeArgIndent}
\begin{flushleft} \textbf{%
\settowidth{\OGTrigFunctionIncludeArgIndent}{OGTrigFunction(}% 
OGTrigFunction(const real \& fx\_  =1., \\ 
\hspace{\OGTrigFunctionIncludeArgIndent}const real \& fy\_  =1., \\ 
\hspace{\OGTrigFunctionIncludeArgIndent}const real \& fz\_  =0., \\ 
\hspace{\OGTrigFunctionIncludeArgIndent}const real \& ft\_  =0., \\ 
\hspace{\OGTrigFunctionIncludeArgIndent}const int \& maximumNumberOfComponents  =10)
}\end{flushleft}
\begin{description}
\item[{\bf Description:}]  
 
 This class is derived from the {\ff OGFunction} class and defines a function
 and defines a function that is a trigonometric polynomial:
 \[
      u_n(x,y,z,t) = a(n) \cos(f_x(n) \pi (x-g_x(n)))
                          \cos(f_y(n) \pi (y-g_y(n)))
                          \cos(f_z(n) \pi (z-g_z(n)))
                          \cos(f_t(n) \pi (t-g_t(n)))  + c(n)
 \]
  where $a(n)$, $f_x(n)$, $f_y(n)$ etc. can be given values for each component n.

\item[{\bf fx\_, fy\_, fz\_, ft\_ (input):}]  give frequencies (constant for all components).
\item[{\bf maximumNumberOfComponents (input):}]  maximum number of components.

\item[{\bf Notes:}] 
    By default $a(n)=1$ and $g_x(n)=g_y(n)=g_z(n)=g_t(n)=0$.
\item[{\bf Author:}]  WDH
\end{description}

 
\begin{flushleft} \textbf{%
\settowidth{\OGTrigFunctionIncludeArgIndent}{OGTrigFunction(}% 
OGTrigFunction(const RealArray \& fx\_, \\ 
\hspace{\OGTrigFunctionIncludeArgIndent}const RealArray \& fy\_, \\ 
\hspace{\OGTrigFunctionIncludeArgIndent}const RealArray \& fz\_, \\ 
\hspace{\OGTrigFunctionIncludeArgIndent}const RealArray \& ft\_)
}\end{flushleft}
\begin{description}
\item[{\bf Description:}]  
 
 Use this constructor to supply different frequencies for different components.

\item[{\bf fx\_, fy\_, fz\_, ft\_ (input):}]  give frequencies for different components. The dimension of fx\_
        will determine the maximumNumberOfComponents.

\end{description}
\subsubsection{setAmplitudes}
 
\begin{flushleft} \textbf{%
int  \\ 
\settowidth{\OGTrigFunctionIncludeArgIndent}{setAmplitudes(}%
setAmplitudes(const RealArray \& a\_ )
}\end{flushleft}
\begin{description}
\item[{\bf Description:}]  
 
 Use this function to supply different amplitudes for different components.

\item[{\bf a\_ (input):}]  give amplitudes for different components. The dimension of a\_
   should be equal to the maximumNumberOfComponents as determined by the call to the constructor.

\end{description}
\subsubsection{setConstants}
 
\begin{flushleft} \textbf{%
int  \\ 
\settowidth{\OGTrigFunctionIncludeArgIndent}{setConstants(}%
setConstants(const RealArray \& c\_ )
}\end{flushleft}
\begin{description}
\item[{\bf Description:}]  
 
 Use this function to supply different constants for different components.

\item[{\bf c\_ (input):}]  give constants for different components. The dimension of c\_
   should be equal to the maximumNumberOfComponents as determined by the call to the constructor.

\end{description}
\subsubsection{setFrequencies}
 
\begin{flushleft} \textbf{%
int  \\ 
\settowidth{\OGTrigFunctionIncludeArgIndent}{setFrequencies(}%
setFrequencies(const RealArray \& fx\_, \\ 
\hspace{\OGTrigFunctionIncludeArgIndent}const RealArray \& fy\_, \\ 
\hspace{\OGTrigFunctionIncludeArgIndent}const RealArray \& fz\_, \\ 
\hspace{\OGTrigFunctionIncludeArgIndent}const RealArray \& ft\_)
}\end{flushleft}
\begin{description}
\item[{\bf Description:}]  
 
 Use this function to supply different frequencies for different components.

\item[{\bf fx\_, fy\_, fz\_, ft\_ (input):}]  give frequencies for different components. The dimension of fx\_
        will determine the maximumNumberOfComponents.

\end{description}
\subsubsection{setShifts}
 
\begin{flushleft} \textbf{%
int  \\ 
\settowidth{\OGTrigFunctionIncludeArgIndent}{setShifts(}%
setShifts(const RealArray \& gx\_, \\ 
\hspace{\OGTrigFunctionIncludeArgIndent}const RealArray \& gy\_, \\ 
\hspace{\OGTrigFunctionIncludeArgIndent}const RealArray \& gz\_, \\ 
\hspace{\OGTrigFunctionIncludeArgIndent}const RealArray \& gt\_)
}\end{flushleft}
\begin{description}
\item[{\bf Description:}]  
 
 Use this function to supply different shifts for different components.

\item[{\bf gx\_, gy\_, gz\_, gt\_ (input):}]  give shifts for different components. The dimensions of gx\_, gy\_,...
   should be equal to the maximumNumberOfComponents as determined by the call to the constructor.

\end{description}
