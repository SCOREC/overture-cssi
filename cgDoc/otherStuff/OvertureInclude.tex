\subsection{Overture global variables.}
 
\newlength{\OvertureIncludeArgIndent}
 
  The Overture class contains global variables that can be used as default arguments
  to functions. For example, {\tt Overture::nullRealArray() }, can be used as a default
  argument for a {\tt RealArray}. These instances of classes are accessed as function calls
  as opposed to building static global variables. Initially the later approach was taken
  but his caused difficulties since the loader would build the classes in some unknown order.

\begin{description}
\item[{\bf floatSerialArray \& nullFloatArray():}] 
\item[{\bf doubleSerialArray \& nullDoubleArray():}] 
\item[{\bf RealArray \& nullRealArray():}] 
\item[{\bf intSerialArray \& nullIntArray():}] 
	 
\item[{\bf floatDistributedArray \& nullFloatDistributedArray():}] 
\item[{\bf doubleDistributedArray \& nullDoubleDistributedArray():}] 
\item[{\bf RealDistributedArray \& nullRealDistributedArray():}] 
\item[{\bf IntegerDistributedArray \& nullIntegerDistributedArray():}] 
	 
\item[{\bf MappingParameters \& nullMappingParameters():}] 
	 
\item[{\bf floatMappedGridFunction \& nullFloatMappedGridFunction():}] 
\item[{\bf doubleMappedGridFunction \& nullDoubleMappedGridFunction():}] 
\item[{\bf realMappedGridFunction \& nullRealMappedGridFunction():}] 
\item[{\bf intMappedGridFunction \& nullIntMappedGridFunction():}] 
	 
\item[{\bf floatGridCollectionFunction \& nullFloatGridCollectionFunction():}] 
\item[{\bf realGridCollectionFunction \& nullRealGridCollectionFunction():}] 
\item[{\bf doubleGridCollectionFunction \& nullDoubleGridCollectionFunction():}] 
\item[{\bf intGridCollectionFunction \& nullIntGridCollectionFunction():}] 
	 
\item[{\bf BoundaryConditionParameters \& defaultBoundaryConditionParameters():}] 
\item[{\bf GraphicsParameters \& defaultGraphicsParameters():}] 

\end{description}
\subsection{start}
 
\begin{flushleft} \textbf{%
int  \\ 
\settowidth{\OvertureIncludeArgIndent}{start(}%
start(int \& argc, char **\&argv)
}\end{flushleft}
\begin{description}
\item[{\bf Description:}] 
    Overture initialization function. Call this routine before calling any Overture functions.

\item[{\bf NOTES:}] 
    In parallel the value of argc and the values in argv will be sent to all processors.
   Thus argc and argv will be changed on all processors except processor 0. 
 
 In parallel call 
     ParallelUtility::broadCastArgsCleanup(argc,argv);
 to delete the argv arrays created by the parallel broad-cast

\end{description}
\subsection{finish}
 
\begin{flushleft} \textbf{%
int  \\ 
\settowidth{\OvertureIncludeArgIndent}{finish(}%
finish()
}\end{flushleft}
\begin{description}
\item[{\bf Description:}] 
    Overture cleanup function. Call this routine when you are done using Overture.
\end{description}
\subsection{getAbortOption}
 
\begin{flushleft} \textbf{%
AbortEnum  \\ 
\settowidth{\OvertureIncludeArgIndent}{getAbortOption(}%
getAbortOption()
}\end{flushleft}
\begin{description}
\item[{\bf Description:}] 
   Return the current abort option.
 \begin{verbatim}
    enum AbortEnum
    {
      abortOnAbort,
      throwErrorOnAbort
    };
 \end{verbatim}
\end{description}
\subsection{setAbortOption}
 
\begin{flushleft} \textbf{%
void  \\ 
\settowidth{\OvertureIncludeArgIndent}{setAbortOption(}%
setAbortOption( AbortEnum action )
}\end{flushleft}
\begin{description}
\item[{\bf Description:}] 
    Specify the action to take when the Overture::abort() function is called.
\end{description}
\subsection{abort}
 
\begin{flushleft} \textbf{%
void  \\ 
\settowidth{\OvertureIncludeArgIndent}{abort(}%
abort()
}\end{flushleft}
\begin{description}
\item[{\bf Description:}] 
    Abort the program or throw an error depending on the value of getAbortOption
\end{description}
\subsection{abort}
 
\begin{flushleft} \textbf{%
void  \\ 
\settowidth{\OvertureIncludeArgIndent}{abort(}%
abort(const aString \& message)
}\end{flushleft}
\begin{description}
\item[{\bf Description:}] 
    Abort the program or throw an error depending on the value of getAbortOption
\item[{\bf message (input) :}]  print this message.
\end{description}
\subsection{getGraphicsInterface}
 
\begin{flushleft} \textbf{%
GenericGraphicsInterface*  \\ 
\settowidth{\OvertureIncludeArgIndent}{getGraphicsInterface(}%
getGraphicsInterface(const aString \& windowTitle/*="Your Slogan Here"*/, const bool initialize/*=true*/,\\ 
\hspace{\OvertureIncludeArgIndent}int argc/* = 0*/, char *argv[] /*=NULL*/)
}\end{flushleft}
\begin{description}
\item[{\bf Description:}] 
   Return a pointer to the one and only graphics interface. If the pointer is null,
   a new graphics interface will be built.
\end{description}
\subsection{setGraphicsInterface}
 
\begin{flushleft} \textbf{%
void  \\ 
\settowidth{\OvertureIncludeArgIndent}{setGraphicsInterface(}%
setGraphicsInterface( GenericGraphicsInterface *ps)
}\end{flushleft}
\begin{description}
\item[{\bf Description:}] 
   Set the default graphics interface. This is normally called by the GenericGraphicsInterface constructor,
 there is no need for a typical user to call this function.
\end{description}
\subsection{getMappingList}
 
\begin{flushleft} \textbf{%
ListOfMappingRC*  \\ 
\settowidth{\OvertureIncludeArgIndent}{getMappingList(}%
getMappingList()
}\end{flushleft}
\begin{description}
\item[{\bf Description:}] 
   Return a pointer to the default list of mappings. This pointer may be NULL.
\end{description}
\subsection{setMappingList}
 
\begin{flushleft} \textbf{%
void  \\ 
\settowidth{\OvertureIncludeArgIndent}{setMappingList(}%
setMappingList(ListOfMappingRC *list)
}\end{flushleft}
\begin{description}
\item[{\bf Description:}] 
   Set the default list of mappings.
\end{description}
\subsection{setDefaultGraphicsParameters}
 
\begin{flushleft} \textbf{%
void  \\ 
\settowidth{\OvertureIncludeArgIndent}{setDefaultGraphicsParameters(}%
setDefaultGraphicsParameters( GraphicsParameters *gp  =NULL)
}\end{flushleft}
\begin{description}
\item[{\bf Description:}] 
   Set the default graphics parameters. By default reset the graphics parameters to
 the standard one.

\end{description}
\subsection{openDebugFile}
 
\begin{flushleft} \textbf{%
void  \\ 
\settowidth{\OvertureIncludeArgIndent}{openDebugFile(}%
openDebugFile()
}\end{flushleft}
\begin{description}
\item[{\bf Description:}] 
    Open the file Overture::debugFile for writing debugging info to.
  On a serial machine the file is named "overture.debug"
  On a parallel machine the file on processor 0 is named "overture.debug"
  while the file on processor X is named "overtureX.debug"
\end{description}
\subsection{checkMemoryUsage}
 
\begin{flushleft} \textbf{%
real  \\ 
\settowidth{\OvertureIncludeArgIndent}{checkMemoryUsage(}%
checkMemoryUsage(const aString \& label, FILE *file  =stdout)
}\end{flushleft}
\begin{description}
\item[{\bf Description:}] 
   Check the current memory usage in Mega-bytes and print a message if the memory use
  has increased by 10 percent. You must first call turnOnMemoryChecking(true) for this
  function to be turned on.

\item[{\bf label (input) :}]  use this label on message
\item[{\bf file (input):}]  output messages to this file. If NULL, output no message.
\item[{\bf Return value:}]  current memory use (Mb).
 
\end{description}
\subsection{getMaximumMemoryUsage}
 
\begin{flushleft} \textbf{%
real  \\ 
\settowidth{\OvertureIncludeArgIndent}{getMaximumMemoryUsage(}%
getMaximumMemoryUsage()
}\end{flushleft}
\begin{description}
\item[{\bf Description:}] 
   Return the maximum memory use recorded by calls to checkMemoryUsage.
 
\item[{\bf Return value:}]  maximum memory use detected (Mb).
 
\end{description}
\subsection{checkMemoryUsage}
 
\begin{flushleft} \textbf{%
real  \\ 
\settowidth{\OvertureIncludeArgIndent}{printMemoryUsage(}%
printMemoryUsage(const aString \& label, FILE *file  =stdout)
}\end{flushleft}
\begin{description}
\item[{\bf Description:}] 
   Display the current memory usage.

\item[{\bf label (input) :}]  use this label on message
\item[{\bf file (input):}]  output messages to this file. If NULL, output no message.
 
\end{description}
\subsection{incrementReferenceCountForPETSc}
 
\begin{flushleft} \textbf{%
int  \\ 
\settowidth{\OvertureIncludeArgIndent}{incrementReferenceCountForPETSc(}%
incrementReferenceCountForPETSc()
}\end{flushleft}
\begin{description}
\item[{\bf Description:}] 
   Increment the reference count for objects that use PETSc.
\end{description}
\subsection{decrementReferenceCountForPETSc}
 
\begin{flushleft} \textbf{%
int  \\ 
\settowidth{\OvertureIncludeArgIndent}{decrementReferenceCountForPETSc(}%
decrementReferenceCountForPETSc()
}\end{flushleft}
\begin{description}
\item[{\bf Description:}] 
   Decrement the reference count for objects that use PETSc.
\end{description}
