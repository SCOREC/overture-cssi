% -*-Mode: LaTeX; -*- 

\newcommand{\LORCO}{ListOfReferenceCountedObjects}
\section{Reference Counted Objects}
\index{reference counting!reference counted objects}

\subsection{Introduction} 

This section describes our notion of reference counted
objects and describes the class {\ff \LORCO}. 

Grids, grid functions and A++ arrays are all reference
counted objects. 

A reference counted object can be
extremely useful for a number of reasons
\begin{itemize}
 \item a reference counted objects can be easily shared
        between different classes without the need to use 
        pointers.
 \item a reference counted object can automatically
        delete itself when it's reference count goes
        to zero. 
 \item Someone who makes a reference to
        a reference counted object does not have to
        worry that the object will be deleted by someone
        else. 
\end{itemize}


Any object can be made into a
reference counted object by supplying certain
member functions and by following certain rules.
We will describe how to do this later in this
section.

A reference counted object extends the
notion of a reference in C++. The statement
\begin{verbatim}
      realArray & u = v;
\end{verbatim}
makes {\ff u} become a reference (alias) for v.
Similiarly the statements
\begin{verbatim}
      realArray u; 
      u.reference(v);
\end{verbatim}
also make u a reference for v; In both cases changing {\ff u}
with a statement like {\ff u=5.;} will cause {\ff v} to also
change.
However, the latter method for creating
a reference is more dynamic. For example, at a later stage {\ff u} can
reference a different array, {\ff u.reference(w)} or {\ff u}
can break the reference {\ff u.breakReference()}. When a reference
is broken {\ff u} will be given it's own copy of the array data.

\subsection{How to Write a Reference Counted Class}

In this section we describe how to write a reference
counted class. Examples of reference counted classes
are {\ff MappedGrid}, {\ff GridCollection}, 
{\ff typeMappedGridFunction} and {\ff typeGridCollectionFunction}.

% The basic philosphy behind writing a reference counted class
% is based on the paradyne?? of a letter and envelope??. The letter
% class holds all the data. 

Here is an example of a header file for a reference counted
class
(file {\ff Overture/examples/Reference\-Counted\-Class.h})
{\footnotesize
\listinginput[1]{1}{ReferenceCountedClass.h}
}

A reference counted class should 
\begin{itemize}
 \item be derived from the {\ff ReferenceCounting} class.
 \item contain a copy constructor which can be a deep or shallow copy.
 \item have an assignment operator which is a deep copy.
 \item have a {\ff reference} function
 \item have a {\ff breakReference} function
 \item create another class to hold all the data associated with the class
       (class RCData in the example).
 \item contain a pointer to a class that holds the data for the object
 \item a private section with functions {\ff reference}, assignment,
       and {\ff virtualConstructor}, see example. These member functions
       allow all class's that are derived from the {\ff ReferenceCounting}
       class to be put in a list.
\end{itemize}

Here is the implementation of {\ff ReferenceCountedClass}
(file {\ff Overture/examples/Reference\-Counted\-Class.C})
{\footnotesize
\listinginput[1]{1}{ReferenceCountedClass.C}
}



A reference counted class should be derived from the 
{\ff ReferenceCounting} class
(file {\ff /Reference\-Counting.h})
{\footnotesize
\listinginput[1]{1}{\grid/ReferenceCounting.h}
}

% Reference Counting
%  This class supports two types of reference counting. The suggested
%  way to do reference counting is to use the reference and breakReference
%  member functions as shown in the examples below. If instead you want
%  to keep pointers to ListOfReferenceObjects then you can use the
%  incrementReferenceCount(), decrementReferenceCount() and
%  getReferenceCount() member functions (derived from the ReferenceCounting
%  in order to manage references). It is up to you to call delete when
%  the reference count reaches zero.


\subsection{Class \LORCO}

This is a template list class for holding holding 
reference counted objects.

The class is declared as
\begin{verbatim}
template<class T>
class ListOfReferenceCountedObjects : public ReferenceCounting
\end{verbatim}
so that {\ff T} is the generic name of the class whose instances
will be placed in the list.

\subsubsection{Constructors}

\begin{tabbing}
{\ff xxxxxxxxxxxxxxxxxxxxxxxxxxxxxxxxxxxxxxxxxxxxxxxxxx} xxxxx \= xxxxxxxxxxxx \=  \kill
{\ff DifferentialAndBoundaryOperators( )} \> Default constructor \\
{\ff ListOfReferenceCountedObjects()}  \> default constructor \\
{\ff ListOfReferenceCountedObjects(int numberOfElements)} \> Create a list with a given number of elements \\
\end{tabbing}

\subsubsection{Public Member Functions}

\noindent
Here are the public member functions. 
\begin{tabbing}
{\ff xxxxxxxxxxxxxxxxxxxxxxxxxxxxxxxxxxxxxxxxxxxxxxxxxxxxxxxxxx} xxxxx \= xxxxxxxxxxx \=  \kill
{\ff ListOfReferenceCountedObjects\& operator=(ListOfReferenceCountedObjects\&)} \> \\
{\ff void addElement(int index)} \> Add an object to the list \\
{\ff void addElement()} \> Add an object to the end of the list \\
{\ff void addElement(T \& t, int index)} \> Add an object and reference to  t \\
{\ff void addElement(T \& t )} \> Add an object to the end, and reference to t \\
{\ff int getLength()} \> Get length of list. \\
{\ff T\& operator[](int index)} \>  Reference the object at a given location. \\
{\ff void deleteElement(T \& X)} \> Find an element on the list and delete it \\
{\ff void deleteElement(int index)} \> Delete the element at the given index \\
{\ff void deleteElement()} \> Delete the element appearing last in the list   \\
{\ff void swapElements(int i, int j)} \>  Swap two elements (for sorting) \\
{\ff int getIndex(T \& X)} \> get the index for an element \\
{\ff void reference(ListOfReferenceCountedObjects<T> \& list )} \> reference one list to another \\
{\ff void breakReference()} \> break any references with this list 
\end{tabbing}


\subsubsection{Examples}

This class supports two types of reference counting. The suggested
way to do reference counting is to use the reference and breakReference
member functions as shown in the examples below. If instead you want
to keep pointers to ListOfReferenceObjects then you can use the
incrementReferenceCount(), decrementReferenceCount() and
getReferenceCount() member functions (derived from the ReferenceCounting
in order to mange references). With this latter mode of reference
counting it is up to you to call delete when
the reference count reaches zero.

Typical Usage:
\begin{verbatim}
     floatArray a(10),b(5);
     ListOfReferenceCountedObjects<floatArray> list,list2;
     list.addElement();      // add an element
     list[0].reference(a);   // reference to array a
     list.addElement(b);     // add element and reference to b in one step
     list[1]=b;
     list[0]=1.;             //  same as a=1.;

     list2.reference(list);  // list2 and list are now the same
     list2[0]=2.;            // same as list[0]=2.;
     list2.breakReference();
     list2[0]=5.;            // does not change list[0]
\end{verbatim}

