\subsection{Grid Generation for a Truck.}

  In this case study we illustrate the use of the hyperbolic grid generator
and mappingBuilder to construct grids on a truck. This `truck' is actually
a wind-tunnel model.


\newcommand{\truckFigures}{\homeHenshaw/Overture/hype/cmd/truck}

% \renewcommand{\figWidth}{.65\linewidth}
% \newcommand{\clipfig}[1]{\psclip{\psframe[linecolor=white](0.,.0)(11.5,10.)}\epsfig{#1}\endpsclip}

{
\newcommand{\figWidthd}{10cm}
\newcommand{\trimfig}[2]{\trimPlot{#1}{#2}{.0}{.0}{.1}{.1}}
\begin{figure}[hbt]
\begin{center}
\begin{tikzpicture}[scale=1]
  \useasboundingbox (0,.5) rectangle (9.,8.5);  % set the bounding box (so we have less surrounding white space)
%
  \draw ( 0.0,0.) node[anchor=south west,xshift=-4pt,yshift=+0pt] {\trimfig{\figures/cabTender}{\figWidthd}};
%
 % \draw (current bounding box.south west) rectangle (current bounding box.north east);
% grid:
%  \draw[step=1cm,gray] (0,0) grid (9,8);
\end{tikzpicture}
\end{center}
\caption{Overlapping grid for the cab of a truck.}\label{fig:cabTender}
\end{figure}
}


%- 
%- \psset{xunit=1.cm,yunit=1.cm,runit=1.cm}
%- \begin{figure}[htb]
%- \begin{center}
%- \begin{pspicture}(0,0.)(12.0,9.)
%- \rput(6.0,4.75){\clipfig{file=\truckFigures/cabTender.ps,width=\figWidth}}
%- % turn on the grid for placement
%- % \psgrid[subgriddiv=2]
%- \end{pspicture}
%- \end{center}
%- \caption{Overlapping grid for the cab of a truck.}\label{fig:cabTender}
%- \end{figure}
%- 
%- 

% ---------------------------------------------------------------------------------------------------------------------
\subsubsection{Front}

  Figure~(\ref{fig:cabFront}) shows the surface grid for the front of the truck.
The starting curve for the grid was generating by cutting the CAD model
with a plane. The surface grid was grown, stretched and smoothed.

{\bf Remarks:}
\begin{itemize}
  \item The equidistribution weight was turned on to generate the surface grid. This
    allowed the grid to march more cleanly over the surface.
  \item The CAD surface grid bends sharply in a small region on top of the
        bumper. The grid was smoothed in this region without projecting onto the CAD surface
      so as to smooth this indentation out a little bit.
\end{itemize}



{
\newcommand{\figWidthd}{7.5cm}
\newcommand{\trimfig}[2]{\trimPlotb{#1}{#2}{.0}{.0}{.0}{.025}}
\begin{figure}[hbt]
\begin{center}
\begin{tikzpicture}[scale=1]
  \useasboundingbox (0,.5) rectangle (16.,15.0); % set the bounding box (so we have less surrounding white space)
%
  \draw ( 0.0,7.5) node[anchor=south west,xshift=-4pt,yshift=+0pt] {\trimfig{\figures/cabFrontCutPlane}{\figWidthd}};
  \draw (8.0,7.5) node[anchor=south west,xshift=-4pt,yshift=+0pt] {\trimfig{\figures/cabFrontInitial}{\figWidthd}};
  \draw (8.0,7.5) node[draw,fill=white,anchor=south,yshift=+0pt] {\small Cut plane is used to generate starting curve.};
  \draw[->,thick,black] (7.8,8.0) -- (2.5,11);
  \draw[->,thick,black] (8.2,8.0) -- (10.7,11);
% \draw (12.0,7.5) node[draw,fill=white,anchor=south,yshift=+0pt] {\parbox{6cm}{\small Stretching is added after marching.}};
%
  \draw (8.0,0.) node[anchor=south west,xshift=-4pt,yshift=+0pt] {\trimfig{\figures/cabFrontSurface}{\figWidthd}};
  \draw (4.0,4.0) node[draw,fill=white,anchor=south,yshift=+0pt] {\parbox{6cm}{\small Stretching is added after marching.}};
  \draw[->,thick,black] (7.1,4.2) -- (9.5,5.4);
  \draw (4.0,0.5) node[draw,fill=white,anchor=south,yshift=+0pt] {\parbox{6cm}{\small Bumper indentation smoothed.}};
  \draw[->,thick,black] (7.1,.7) -- (9.5,3.2);
%
 % \draw (current bounding box.south west) rectangle (current bounding box.north east);
% grid:
%  \draw[step=1cm,gray] (0,0) grid (16,15);
\end{tikzpicture}
\end{center}
\caption{Grids for the front of the cab. Top left: starting curve is generated by intersecting a plane with the surface.
    Top right: surface grid is grown. Bottom: surface grid is stretched and smoothed. The
  identation over the bumper is smoothed slightly by locally smoothing without projecting
  onto the CAD surface.}
\label{fig:cabFront}
\end{figure}
}



%- \renewcommand{\figWidth}{.45\linewidth}
%- % \renewcommand{\clipfig}[1]{\psclip{\psframe[linewidth=2pt](0.,.6)(5.7,5.6)}\epsfig{#1}\endpsclip}
%- \renewcommand{\clipfig}[1]{\psclip{\psframe[linecolor=white](0.,.0)(8.1,8.2)}\epsfig{#1}\endpsclip}
%- \newcommand{\clipfigb}[1]{\psclip{\psframe[linecolor=white](0.,1.)(8.1,8.2)}\epsfig{#1}\endpsclip}
%- 
%- \psset{xunit=1.cm,yunit=1.cm,runit=1.cm}
%- %%BoundingBox: 36 36 576 576 
%- \begin{figure}[htb]
%- \begin{center}
%- \begin{pspicture}(0,-7.)(17.5,8.)
%- \rput( 4.5, 3.75){\clipfig{file=\truckFigures/cabFrontCutPlane.ps,width=\figWidth}}
%- \rput(13.25, 3.75){\clipfig{file=\truckFigures/cabFrontInitial.ps,width=\figWidth}}
%- \rput(13.25,-4.20){\clipfigb{file=\truckFigures/cabFrontSurface.ps,width=\figWidth}}
%- % turn on the grid for placement
%- % \psgrid[subgriddiv=2]
%- \rput*(.5,-1.6){\makebox(0,0)[l]{Cut plane is used to generate starting curve.}}
%- \psline[linewidth=1.pt]{->}(2.,-1.5)(3.,1.)
%- \psline[linewidth=1.pt]{->}(7.25,-1.45)(12.1,3.)
%- \rput*(8.5,-3.5){\makebox(0,0)[r]{Stretching is added after marching.}}
%- \psline[linewidth=1.pt]{->}(8.6,-3.5)(10.6,-2.75)
%- \rput*(8.5,-4.5){\makebox(0,0)[r]{Bumper indentation smoothed.}}
%- % \psline[linewidth=1.pt]{->}(8.5,-4.5)(5.,2.5)
%- \psline[linewidth=1.pt]{->}(8.6,-4.5)(11.,-4.75)
%- %\rput(9.5, 5){\makebox(0,0)[c]{\small Pressure, $t=1.75$}}
%- \end{pspicture}
%- \end{center}
%- \caption{Grids for the front of the cab. Top left: starting curve is generated by intersecting a plane with the surface.
%-     Top right: surface grid is grown. Bottom: surface grid is stretched and smoothed. The
%-   identation over the bumper is smoothed slightly by locally smoothing without projecting
%-   onto the CAD surface.}
%- \label{fig:cabFront}
%- \end{figure}

\clearpage
% --------------------------------------------------------------------------------------------------------------------
\subsubsection{Wheel}

Each wheel is covered with two grids as shown in figure~(\ref{fig:frontWheel}).
The tricky part here is to have a grid line follow all the corners. 

{\bf Remarks:}
\begin{itemize}
  \item The surface grid for the wheel-body join was generated by matching to
      `interior matching curves'. This causes grid lines to follow the edges of the wheel.
\end{itemize}


{
\newcommand{\figWidthd}{7.5cm}
\newcommand{\trimfig}[2]{\trimPlot{#1}{#2}{.0}{.0}{.0}{.025}}
\begin{figure}[hbt]
\begin{center}
\begin{tikzpicture}[scale=1]
  \useasboundingbox (0,.5) rectangle (16.,15.0); % set the bounding box (so we have less surrounding white space)
%
  \draw ( 0.0,7.5) node[anchor=south west,xshift=-4pt,yshift=+0pt] {\trimfig{\figures/frontWheelSurfaceInitial}{\figWidthd}};
  \draw[->,thick,black] (4,13.5) -- (4,12.4);
  \draw (4.0,13.5) node[draw,fill=white,anchor=south,yshift=+0pt] {\small Starting curve.};
  \draw[->,thick,black] (3.8,8.2) -- (1.4,11.2);
  \draw[->,thick,black] (4.2,8.2) -- (3.3,11.0);
  \draw (4.0,8.0) node[draw,fill=white,anchor=south,yshift=+0pt] {\small Grid is grown matching to interior curves.};
% 
  \draw (8.0,7.5) node[anchor=south west,xshift=-4pt,yshift=+0pt] {\trimfig{\figures/frontRightWheelJoinSurface}{\figWidthd}};
  \draw[->,thick,black] (12.,8.2) -- (10.7,11.4);
  \draw (12.0,8.0) node[draw,fill=white,anchor=south,yshift=+0pt] {\small Stretching added.};
% \draw (12.0,7.5) node[draw,fill=white,anchor=south,yshift=+0pt] {\parbox{6cm}{\small Stretching is added after marching.}};
%
  \draw (0.0,0.) node[anchor=south west,xshift=-4pt,yshift=+0pt] {\trimfig{\figures/frontRightWheelJoin}{\figWidthd}};
  \draw (8.0,0.) node[anchor=south west,xshift=-4pt,yshift=+0pt] {\trimfig{\figures/frontRightWheelSurface}{\figWidthd}};
% 
  \draw[->,thick,black] (4.0,.7) -- (2.0,4.5);
  \draw (4.0,.05) node[draw,fill=white,anchor=south,yshift=+0pt] {\parbox{6cm}{\small Volume grid grown and matched to body surface.}};
%
  \draw[->,thick,black] (12,.7) -- (11.6,2.7);
  \draw (12.0,0.5) node[draw,fill=white,anchor=south,yshift=+0pt] {{\small Starting curve.}};
%
 % \draw (current bounding box.south west) rectangle (current bounding box.north east);
% grid:
%  \draw[step=1cm,gray] (0,0) grid (16,15);
\end{tikzpicture}
\end{center}
\caption{Wheel Grids. Top left: Surface grid is generated by matching to `interior matching curves'.
    Top right: surface grid is stretched and smoothed.
  Bottom left: volume grid for wheel-body-join is grown by matching to body surface.
Bottom right: Surface grid for wheel.}
\label{fig:frontWheel}
\end{figure}
}

%- 
%- 
%- 
%- \renewcommand{\clipfig}[1]{\psclip{\psframe[linecolor=white](0.,1.25)(8.1,7.)}\epsfig{#1}\endpsclip}
%- \renewcommand{\clipfigb}[1]{\psclip{\psframe[linecolor=white](0.,0.)(8.1,6)}\epsfig{#1}\endpsclip}
%- 
%- \psset{xunit=1.cm,yunit=1.cm,runit=1.cm}
%- %%BoundingBox: 36 36 576 576 
%- \begin{figure}[htb]
%- \begin{center}
%- \begin{pspicture}(0,-6.)(17.5,7)
%- \rput( 4.5, 3.75){\clipfig{file=\truckFigures/frontWheelSurfaceInitial.ps,width=\figWidth}}
%- \rput(13.25, 3.75){\clipfig{file=\truckFigures/frontRightWheelJoinSurface.ps,width=\figWidth}}
%- \rput( 4.5 ,-2.20){\clipfigb{file=\truckFigures/frontRightWheelJoin.ps,width=\figWidth}}
%- \rput(13.25,-3.20){\clipfig{file=\truckFigures/frontRightWheelSurface.ps,width=\figWidth}}
%- % turn on the grid for placement
%- % \psgrid[subgriddiv=2]
%- \rput*(.5,6.5){\makebox(0,0)[l]{Starting curve.}}
%- \psline[linewidth=1.pt]{->}(3.,6.4)(4.5,4.75)
%- \rput*(.5,.7){\makebox(0,0)[l]{Grid is grown matching to interior curves.}}
%- \psline[linewidth=1.pt]{->}(3.,1)(3.75,3.75)
%- \psline[linewidth=1.pt]{->}(3.,1)(1.75,3.75)
%- \rput*(10.5,.7){\makebox(0,0)[l]{Stretching added.}}
%- \psline[linewidth=1.pt]{->}(12.,1)(11.75,4.25)
%- %
%- \rput*(.5,-6.){\makebox(0,0)[l]{Volume grid grown and matched to body surface.}}
%- \psline[linewidth=1.pt]{->}(3.,-5.9)(3.,-1.5)
%- %
%- \rput*(13.5,.2){\makebox(0,0)[l]{Starting curve.}}
%- \psline[linewidth=1.pt]{->}(14.,.1)(13.,-4.5)
%- % \psline[linewidth=1.pt]{->}(7.25,-1.45)(12.1,3.)
%- % \rput*(8.5,-3.5){\makebox(0,0)[r]{Stretching is added after marching.}}
%- % \psline[linewidth=1.pt]{->}(8.6,-3.5)(10.6,-2.75)
%- % \rput*(8.5,-4.5){\makebox(0,0)[r]{Bumper indentation smoothed.}}
%- % % \psline[linewidth=1.pt]{->}(8.5,-4.5)(5.,2.5)
%- % \psline[linewidth=1.pt]{->}(8.6,-4.5)(11.,-4.75)
%- %\rput(9.5, 5){\makebox(0,0)[c]{\small Pressure, $t=1.75$}}
%- \end{pspicture}
%- \end{center}
%- \caption{Wheel Grids. Top left: Surface grid is generated by matching to `interior matching curves'.
%-     Top right: surface grid is stretched and smoothed.
%-   Bottom left: volume grid for wheel-body-join is grown by matching to body surface.
%- Bottom right: Surface grid for wheel.}
%- \label{fig:frontWheel}
%- \end{figure}
%- 

\clearpage
% -------------------------------------------------------------------------------------------------------------------
\subsubsection{Cab tender}

  The grid for the cab tender was built by projecting a transfinite interpolation (TFI) mapping
onto the CAD surface. This option is available from the {\bf MappingBuilder}.
This gave a nicer grid than the hyperbolic grid generator.
Two curves were defined for the TFI mapping by intersecting the CAD surface with planes.

{
\newcommand{\figWidthd}{7.5cm}
\newcommand{\trimfig}[2]{\trimPlot{#1}{#2}{.0}{.0}{.0}{.025}}
\begin{figure}[hbt]
\begin{center}
\begin{tikzpicture}[scale=1]
  \useasboundingbox (0,.5) rectangle (16.,15.0); % set the bounding box (so we have less surrounding white space)
%
  \draw ( 0.0,7.5) node[anchor=south west,xshift=-4pt,yshift=+0pt] {\trimfig{\figures/tenderBackCutPlane}{\figWidthd}};
  \draw[->,thick,black] (4,8.2) -- (6.,12.);
  \draw (4.0,8.0) node[draw,fill=white,anchor=south,yshift=+0pt] {\small Plane cuts surface to create end curve for TFI.};
% 
  \draw (8.0,7.5) node[anchor=south west,xshift=-4pt,yshift=+0pt] {\trimfig{\figures/tenderTFISurface}{\figWidthd}};
% 
  \draw[->,thick,black] (10.5,14.0) -- (9.4,12.7);
  \draw[->,thick,black] (13,14.0) -- (13.8,11.2);
  \draw (12.0,14.0) node[draw,fill=white,anchor=south,yshift=+0pt] {\small TFI built from two end curves.};
  \draw (12.0,8.0) node[draw,fill=white,anchor=south,yshift=+0pt] {\small TFI before projecting onto the surface.};
%  \draw[->,thick,black] (12.,8.2) -- (10.7,11.4);
%   \draw (12.0,8.0) node[draw,fill=white,anchor=south,yshift=+0pt] {\small Stretching added.};
% \draw (12.0,7.5) node[draw,fill=white,anchor=south,yshift=+0pt] {\parbox{6cm}{\small Stretching is added after marching.}};
%
  \draw (0.0,0.) node[anchor=south west,xshift=-4pt,yshift=+0pt] {\trimfig{\figures/tenderSurface}{\figWidthd}};
  \draw (8.0,0.) node[anchor=south west,xshift=-4pt,yshift=+0pt] {\trimfig{\figures/tenderVolume}{\figWidthd}};
% 
  \draw[->,thick,black] (4.0,.5) -- (4.7,3.5);
  \draw (4.0,.05) node[draw,fill=white,anchor=south,yshift=+0pt] {{\small Projected and stretched TFI.}};
  \draw (12.0,.05) node[draw,fill=white,anchor=south,yshift=+0pt] {{\small Volume grid.}};
%
%   \draw[->,thick,black] (12,.7) -- (11.6,2.7);
%   \draw (12.0,0.5) node[draw,fill=white,anchor=south,yshift=+0pt] {{\small Starting curve.}};
%
 % \draw (current bounding box.south west) rectangle (current bounding box.north east);
% grid:
%  \draw[step=1cm,gray] (0,0) grid (16,15);
\end{tikzpicture}
\end{center}
\caption{Cab Tender Grids. }
\label{fig:cabTender}
\end{figure}
}

%- 
%- \renewcommand{\clipfig}[1]{\psclip{\psframe[linecolor=white](0.,1.)(7.9,7.)}\epsfig{#1}\endpsclip}
%- 
%- \psset{xunit=1.cm,yunit=1.cm,runit=1.cm}
%- %%BoundingBox: 36 36 576 576 
%- \begin{figure}[htb]
%- \begin{center}
%- \begin{pspicture}(.25,-6.)(17.,7)
%- \rput( 4.5, 3.75){\clipfig{file=\truckFigures/tenderBackCutPlane.ps,width=\figWidth}}
%- \rput(13.25, 3.75){\clipfig{file=\truckFigures/tenderTFISurface.ps,width=\figWidth}}
%- \rput( 4.5 ,-2.80){\clipfig{file=\truckFigures/tenderSurface.ps,width=\figWidth}}
%- \rput(13.25,-2.90){\clipfig{file=\truckFigures/tenderVolume.ps,width=\figWidth}}
%- % turn on the grid for placement
%- % \psgrid[subgriddiv=2]
%- \rput*(.5,.6){\makebox(0,0)[l]{Plane cuts surface to create end curve for TFI.}}
%- \psline[linewidth=1.pt]{->}(5.,.8)(7.,4.)
%- % \psline[linewidth=1.pt]{->}(3.,1)(1.75,3.75)
%- \rput*(10.5,6.5){\makebox(0,0)[l]{TFI built from two end curves.}}
%- \psline[linewidth=1.pt]{->}(11.,6)(11.,5.25)
%- \psline[linewidth=1.pt]{->}(14.5,6)(15.5,3.5)
%- \rput*(10.5,.6){\makebox(0,0)[l]{TFI before projecting onto the surface.}}
%- \psline[linewidth=1.pt]{->}(12.,1)(11.75,2.25)
%- \rput*(4.5,-1.){\makebox(0,0)[l]{Projected and stretched TFI.}}
%- \psline[linewidth=1.pt]{->}(6.,-1.1)(5.5,-3.)
%- \end{pspicture}
%- \end{center}
%- \caption{Cab Tender Grids. }
%- \label{fig:cabTender}
%- \end{figure}
%- 