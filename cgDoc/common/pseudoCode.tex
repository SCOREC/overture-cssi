%
%  Pseudo-code for CG routines
%
\documentclass[11pt]{article}
% \usepackage[bookmarks=true]{hyperref}  % this changes the page location !
\usepackage[bookmarks=true,colorlinks=true,linkcolor=blue]{hyperref}

% \input documentationPageSize.tex
\hbadness=10000 
\sloppy \hfuzz=30pt

% \voffset=-.25truein
% \hoffset=-1.25truein
% \setlength{\textwidth}{7in}      % page width
% \setlength{\textheight}{9.5in}    % page height

\usepackage{calc}
\usepackage[lmargin=.75in,rmargin=.75in,tmargin=.75in,bmargin=.75in]{geometry}

% \input homeHenshaw

% \input{pstricks}\input{pst-node}
% \input{colours}
\newcommand{\blue}{\color{blue}}
\newcommand{\green}{\color{green}}
\newcommand{\red}{\color{red}}
\newcommand{\black}{\color{black}}


\newcommand{\Largebf}{\sffamily\bfseries\Large}
\newcommand{\largebf}{\sffamily\bfseries\large}
\newcommand{\largess}{\sffamily\large}
\newcommand{\Largess}{\sffamily\Large}
\newcommand{\bfss}{\sffamily\bfseries}
\newcommand{\smallss}{\sffamily\small}

% --------------------------------------------
% \input{pstricks}\input{pst-node}
% \input{colours}

% or use the epsfig package if you prefer to use the old commands
% \usepackage{epsfig}
% \usepackage{calc}
% \input clipFig.tex

% The amssymb package provides various useful mathematical symbols
\usepackage{amsmath}
\usepackage{amssymb}

\input ../common/wdhDefinitions.tex
\newcommand{\bni}{\bigskip\noindent}
\newcommand{\mni}{\medskip\noindent}
\newcommand{\sni}{\smallskip\noindent}
\newcommand{\Dc}{{\mathcal D}}
\newcommand{\Kc}{{\mathcal K}}
% \input ../common/defs.tex

\usepackage{enumitem}% to reduce separation in itemize

\newcommand{\tableFont}{\footnotesize}% font size for tables

% \usepackage{verbatim}
% \usepackage{moreverb}
% \usepackage{graphics}    
% \usepackage{epsfig}    
% \usepackage{fancybox}    

% tell TeX that is ok to have more floats/tables at the top, bottom and total
\setcounter{bottomnumber}{5} % default 2
\setcounter{topnumber}{5}    % default 1 
\setcounter{totalnumber}{10}  % default 3
\renewcommand{\textfraction}{.001}  % default .2

\begin{document}
 
\title{Pseudo-code for CG routines and algorithms}

\author{
William D. Henshaw,\\
 ~\\
Department of Mathematical Sciences, \\
Rensselaer Polytechnic Institute, \\
Troy, NY, USA, 12180.
% Centre for Applied Scientific Computing, \\
% Lawrence Livermore National Laboratory, \\
% henshaw@llnl.gov 
}
 
\maketitle

\tableofcontents

% 
\newcommand{\bc}[1]{\mbox{\bfss#1}}   % bold name
\newcommand{\cc}[1]{\mbox{$//$  #1}}  % comment
%
\newcommand{\ia}{\quad}        % indent level 1
\newcommand{\ib}{\ia\quad}     % indent level 2
\newcommand{\ic}{\ib\quad}     % indent level 3
\newcommand{\id}{\ic\quad}     % indent level 4
\newcommand{\ie}{\id\quad}     % indent level 5

\newcommand{\FUNC}[1]{{\blue#1}}
\newcommand{\IF}{{\blue if}}
\newcommand{\ELSEIF}{{\blue else if}}
\newcommand{\FOR}{{\blue for}}
\newcommand{\END}{{\blue end}}
\newcommand{\ForDomain}{{\blue ForDomain}}
\newcommand{\COM}[1]{{\red\em #1}}

% ------------------------------------------------------------------------
\clearpage 
\section{Time-stepping pseudo code}

% ---------------------------------------------------------------------------------------------------------------------
\subsection{DomainSolver::advanceAdamsPredictorCorrector}\label{sec:DomainSolver::advanceAdamsPredictorCorrector}

Here is an overview of the {\tt DomainSolver::advanceAdamsPredictorCorrector} function (cg/common/src/advancePC.bC)
This is an explicit Adams predictor-corrector time stepper.
\begin{figure}[hbt]%
\begin{flushleft}\tt\small
DomainSolver::\FUNC{advanceAdamsPredictorCorrector}( ..., numberOfSubSteps, ..) \\
\{  \\
\ia initialize \\
\ia \FOR( int mst=1; mst$\le$numberOfSubSteps; mst++ ) \COM{ take time steps} \\
\ib   \IF {\em adapt grids} \\
\ic   adaptGrids( ... ) \\
\ib   \IF {\em move grids} \\
\ic     moveGrids( ... );  \COM{(Sec.~\ref{sec:DomainSolver::moveGrids})}\\
\ic     exposedPoints.interpolate(...) \\
\ib \\
\ib   \COM{ predictor step:} \\
\ib   getUt( ... ) \\
\ib   interpolateAndApplyBoundaryConditions( ... ); \\
\ib   solveForTimeIndependentVariables( ... ) \\
\ib   correctMovingGrids( ... ) \\
\ib \\ 
\ib   \FOR( int correction=0; correction$<$numberOfCorrections; correction++ ) \\
\ic     \COM{ corrector step:} \\
\ic     getUt( ... ) \\
\ic     solveForTimeIndependentVariables( ... ) \\
\ic     interpolateAndApplyBoundaryConditions( ... ); \\
\ic     correctMovingGrids( ... ) \\
\}
\end{flushleft}
\caption{Pseudo-code outline of the advanceAdamsPredictorCorrector function.}
\label{alg:advancePC}
\end{figure}

% ------------------------------------------------------------------------
\clearpage 
\section{Moving grid  pseudo code}

% ---------------------------------------------------------------------------------------------------------------------
\subsection{DomainSolver::moveGrids}\label{sec:DomainSolver::moveGrids}

Pseudo-code for {\tt DomainSolver::moveGrids} (cg/common/src/move.C)
\begin{flushleft}\tt\small
DomainSolver::\FUNC{moveGrids}( t1,t2,t3,dt0, cgf1,cgf2,cgf3 ) \\
\{  \\
\ia setInterfacesAtPastTimes( t1,t2,t3,dt0,cgf1,cgf2,cgf3 ); {\em initialize interfaces} \\
\ia parameters.dbase.get<MovingGrids>("movingGrids").moveGrids(t1,t2,t3,dt0,cgf1,cgf2,cgf3 ); \\
\ia gridGenerator->updateOverlap( cg, mapInfo ); \COM{ regenerate the grid with Ogen} \\
\}
\end{flushleft}

% ---------------------------------------------------------------------------------------------------------------------
\subsection{MovingGrids::moveGrids}\label{sec:MovingGrids::moveGrids}

Pseudo-code for {\tt MovingGrids::moveGrids} (cg/common/moving/src/MovingGrids.C)
\begin{flushleft}\tt
MovingGrids::\FUNC{moveGrids}( t1,t2,t3,dt0, cgf1,cgf2,cgf3 ) \\
\{  \\
\ia  \COM{First move the bodies (but not the grids):}  \\
\ia  detectCollisions(cgf1); \\
\ia  rigidBodyMotion( t1,t2,t3,dt0,cgf1,cgf2,cgf3 );\\
\ia  moveDeformingBodies( t1,t2,t3,dt0,cgf1,cgf2,cgf3 );\\
\ia  userDefinedMotion( t1,t2,t3,dt0,cgf1,cgf2,cgf3 );\\
\ia  \COM{Apply any matrix motions: rotate, shift, scale} \\
\ia \\
\ia  getGridVelocity( cgf2,t2 ); \\
\ia  \COM{Now move the grids:}  \\
\ia  \FOR( grid=0; grid<numberOfBaseGrids; grid++ )  \\
\ib     MatrixTransform \& transform = *cgf3.transform[grid];  \\
\ib     \IF( moveOption(grid)==matrixMotion ) \\
\ic       \COM{apply specified rotation and/or shift}  \\
\ic       transform.rotate(...)  \\
\ib     \ELSEIF( moveOption(grid)==rigidBody ) \\
\ic        \COM{rotate and shift the rigid body}        \\
\ic       transform.rotate(...)  \\
\ia  \FOR( int b=0; b<numberOfDeformingBodies; b++ ) \\
\ib     deformingBodyList[b]->regenerateComponentGrids(  newT, cgf3.cg );   \\
\ia   \\
\ia  getGridVelocity( cgf3,t3 );  \\
\}
\end{flushleft}

% ---------------------------------------------------------------------------------------------------------------------
\subsection{MovingGrids::moveDeformingBodies}\label{sec:MovingGrids::moveDeformingBodies}

Pseudo-code for {\tt MovingGrids::moveDeformingBodies} (cg/common/moving/src/MovingGrids.C)
\begin{flushleft}\tt
MovingGrids::\FUNC{moveDeformingBodies}( t1,t2,t3,dt0, cgf1,cgf2,cgf3 ) \\
\{  \\
\ia  \FOR( int b=0; b<numberOfDeformingBodies; b++ ) \\
\ib    deformingBodyList[b]->integrate( t1,t2,t3,cgf1,cgf2,cgf3, stress); \\
\}
\end{flushleft}

% ===========================================================================================
\clearpage 
\section{DeformingBodyMotion pseudo code}

The {\tt DeformingBodyMotion} class handles deforming bodies. 



% ---------------------------------------------------------------------------------------------------------------------
\subsection{DeformingBodyMotion::integrate}\label{sec:DeformingBodyMotion::integrate}

Pseudo-code for {\tt DeformingBodyMotion::integrate} (cg/common/moving/src/DeformingBodyMotion.C)
This function is called by {\tt MovingGrids::movingGrids} to move the deforming body (but not the
grid associated with the deforming body). 
\begin{flushleft}\tt
MovingGrids::\FUNC{integrate}( t1,t2,t3,dt0, cgf1,cgf2,cgf3, stress ) \\
\{  \\
\ia \IF( elasticShell )\\
\ib    advanceElasticShell(t1,t2,t3,cgf1,cgf2,cgf3,stress,option); \\
\ia \ELSEIF( ... ) \\
\ia \\
\ia \FOR( int face=0; face<numberOfFaces; face++ ) \\
\ib   \IF( ... ) \\
\ib   \ELSEIF( userDefinedDeformingBodyMotionOption==interfaceDeform )  \\
\ic     \COM{The deformed surface is obtained from the boundaryData array:} \\
\ic     RealArray \& bd = parameters.getBoundaryData(side,axis,grid,cg[grid]); \\
\ic     x0 = bd; \\
\ib \\
\}
\end{flushleft}

% ---------------------------------------------------------------------------------------------------------------------
\subsection{DeformingBodyMotion::regenerateComponentGrids}\label{sec:DeformingBodyMotion::regenerateComponentGrids}

Pseudo-code for {\tt DeformingBodyMotion::regenerateComponentGrids} (cg/common/moving/src/DeformingBodyMotion.C)
This function is called by {\tt MovingGrids::movingGrids} (after calling {\tt DeformingBodyMotion:;integrate})
to actually generate the grid associated with the
deforming body.
\begin{flushleft}\tt
DeformingBodyMotion::\FUNC{regenerateComponentGrids}( const real newT, CompositeGrid \& cg) \\
\{  \\
\ia \FOR( int face=0; face<numberOfFaces; face++ ) \\
\ib    hyp.generate();  \COM{Call the hyperbolic grid generator.} \\
\ib    \COM{Save the grid in the GridEvolution list:} \\
\ib    gridEvolution[face]->addGrid(dpm.getDataPoints(),newT); \\
\}
\end{flushleft}

% ---------------------------------------------------------------------------------------------------------------------
\subsection{DeformingBodyMotion::correct}\label{sec:DeformingBodyMotion::correct}

Pseudo-code for {\tt DeformingBodyMotion::correct} (cg/common/moving/src/DeformingBodyMotion.C)
This function is called by {\tt MovingGrids::correctGrids}.
\begin{flushleft}\tt
DeformingBodyMotion::\FUNC{correct}( t1, t2, GridFunction \& cgf1,GridFunction \& cgf2 ) \\
\{  \\
\ia \COM{This function currently does nothing.} \\
\}
\end{flushleft}



% ------------------------------------------------------------------------
\clearpage 

\section{Cgmp}

\input cgmpPseudoCode

% % ---------------------------------------------------------------------------------------------------
% \subsection{Cgmp::solve}
% Pseudo-code for {\tt Cgmp::solve} (cg/mp/src/solve.C)

% \begin{flushleft}\tt\small
% Cgmp::\FUNC{solve}() \\
% \{  \\
% \ia  cycleZero(); \\
% \ia  buildRunTimeDialog(); \\
% \ia  \FOR( int step=0; step$<$maximumNumberOfSteps \&\& !finish;  ) \\
% \ib    \IF( t$\ge$ nextTimeToPrint ) \\
% \ic      printTimeStepInfo(step,t,cpuTime); \\
% \ic      saveShow( gf[current] );  \\
% \ic      finish=plot(t, optionIn, tFinal);   \\
% \ic      \IF( finish ) break; \\
% \ib    dtNew = getTimeStep( gf[current] ); \COM{choose time step}  \\
% \ib    computeNumberOfStepsAndAdjustTheTimeStep(t,tFinal,nextTimeToPrint,numberOfSubSteps,dtNew); \\
% \ib \\
% \ib    advance(tFinal); \COM{advance to t=nextTimeToPrint}  \\
% \ib  \\
% \}
% \end{flushleft}


% \clearpage
% % ---------------------------------------------------------------------------------------------------
% \subsection{Cgmp::multiDomainAdvance}\label{sec:Cgmp::multiDomainAdvance}

% Pseudo-code for {\tt Cgmp::multiDomainAdvance} (cg/mp/src/multiDomainAdvance.C)


% \begin{flushleft}\tt\small
% Cgmp::\FUNC{multiDomainAdvance}( real \&t, real \& tFinal ) \\
% \{  \\
% \ia  \IF( initialize ) \\
% \ib    initializeInterfaceBoundaryConditions( t,dt,gfIndex );\\
% \ib    ForDomain( d ) assignInterfaceRightHandSide( d, t, dt, correct, gfIndex );\\
% \ib    ForDomain( d ) domainSolver[d]->initializeTimeStepping( t,dt );\\
% \ia  \COM{Take some time steps:}\\
% \ia  \FOR( int i=0; i<numberOfSubSteps; i++ )\\
% \ib    ForDomain( d )\\
% \ic      domainSolver[d]->startTimeStep( t,dt,... );\\
% \ic      numberOfRequiredCorrectorSteps=...; gridHasChanged=...; \\ 
% \ib    \IF( gridHasChanged ) \\
% \ic     initializeInterfaces(gfIndex); initializeInterfaceBoundaryConditions(...); \\
% \ib \\
% \ib    \FOR( int correct=0; correct<=numberOfCorrectorSteps; correct++ ) \\
% \ic       ForDomain( d )\\
% \id        assignInterfaceRightHandSide( d, t+dt, dt, correct, gfIndex ); \COM{(Sec.~\ref{sec:Cgmp::assignInterfaceRightHandSide})} \\
% \id        domainSolver[d]->takeTimeStep( t,dt,correct,advanceOptions[d] );  \\
% \ic      \IF( hasConverged = checkInterfaceForConvergence( .. ) ) break; \\
% \ib \\
% \ib    ForDomain( d ) domainSolver[d]->endTimeStep( td,dt,advanceOptions[d] ); \\
% \ib    t+=dt;  \\
% \}
% \end{flushleft}

% \clearpage
% % ---------------------------------------------------------------------------------------------------
% \subsection{Cgmp::multiDomainAdvanceNew}\label{sec:Cgmp::multiDomainAdvanceNew}

% Pseudo-code for {\tt Cgmp::multiDomainAdvanceNew} (cg/mp/src/multiDomainAdvanceNew.bC). This is 
% the new versio of the multi-domain advance routine that supports more general time stepping and
% the use of AMR.


% \begin{flushleft}\tt\small
% Cgmp::\FUNC{multiDomainAdvanceNew}( real \&t, real \& tFinal ) \\
% \{  \\
% \ia  \IF( initialize ) \\
% \ib    initializeInterfaceBoundaryConditions( t,dt,gfIndex );\\
% \ib    ForDomain( d ) assignInterfaceRightHandSide( d, t, dt, correct, gfIndex );\\
% \ib    ForDomain( d ) domainSolver[d]->initializeTimeStepping( t,dt );\\
% \ia  \COM{Take some time steps:}\\
% \ia  \FOR( int i=0; i<numberOfSubSteps; i++ )\\
% \ib    ForDomain( d )\\
% \ic      domainSolver[d]->startTimeStep( t,dt,... );\\
% \ic      numberOfRequiredCorrectorSteps=...; gridHasChanged=...; \\ 
% \ib    \IF( gridHasChanged ) \\
% \ic      initializeInterfaces(gfIndex); initializeInterfaceBoundaryConditions(...); \\
% \ib    \COM{Get current interface residual and save current interface values :} \\
% \ib    getInterfaceResiduals( t, dt, gfIndex, maxResidual, saveInterfaceTimeHistoryValues ); \\
% \ib \\
% \ic       \COM{Stage I: advance the solution but do not apply BC's:} \\
% \ic       ForDomain( d )  \\
% \id         assignInterfaceRightHandSide( d, t+dt, dt, correct, gfIndex ); \\
% % \id         advanceOptions[d].takeTimeStepOption=takeStepButDoNotApplyBoundaryConditions; \\
% \id         domainSolver[d]->takeTimeStep( t,dt,correct,{\em\green step but no BC's} );  \\
% \ic \\
% \ic       \COM{Stage II: Project interface values part 1:} \\
% \ic       interfaceProjection( t+dt, dt, correct, gfIndex,{\em\green set interface values} ); \\
% \ic \\
% \ic       \COM{Stage III: evaluate the interface conditions and apply the boundary conditions:} \\
% \ic       ForDomain( d )\\
% \id        assignInterfaceRightHandSide( d, t+dt, dt, correct, gfIndex ); \COM{(Sec.~\ref{sec:Cgmp::assignInterfaceRightHandSide})} \\
% % \id        advanceOptions[d].takeTimeStepOption=applyBoundaryConditionsOnly; \\
% \id        domainSolver[d]->takeTimeStep( t,dt,correct,{\em\green apply BC's} );  \\
% \ic \\
% \ic       \COM{Stage IV: Project interface values part 2:} \\
% \ic       interfaceProjection( t+dt, dt, correct, gfIndex,{\em\green set interface ghost values} ); \\
% \ic \\
% \ic      \IF( hasConverged = checkInterfaceForConvergence( .. ) ) break; \\
% \ib \\
% \ib    ForDomain( d ) domainSolver[d]->endTimeStep( td,dt,advanceOptions[d] ); \\
% \ib    t+=dt;  \\
% \}
% \end{flushleft}

% % -------------------------------------------------------------------------
% \clearpage
% \section{Interfaces }

% % ---------------------------------------------------------------------------------------------------------
% \subsection{Cgmp::assignInterfaceRightHandSide} \label{sec:Cgmp::assignInterfaceRightHandSide}

% The {\tt Cgmp::assignInterfaceRightHandSide} function is used to get interface values from a source
% domain and set interface values on a target domain. It is used in the Cgmp::multiDomainAdvance routine~\ref{sec:Cgmp::multiDomainAdvance}.


% Here is {\tt Cgmp::assignInterfaceRightHandSide} (cg/mp/src/assignInterfaceBoundaryConditions.C)
% \begin{flushleft}\tt\small
% int Cgmp::\FUNC{assignInterfaceRightHandSide}( int d, real t, real dt, int correct, std::vector<int> \& gfIndex ) \\
% \COM{d : target domain, assign the interface RHS for this domain.} \\
% \{  \\
% \ia \IF( interfaceList.size()==0 ) \\
% \ib     initializeInterfaces(gfIndex); \COM{Create the list of interfaces.} \\
% \ib \\
% \ia \FOR( each interface on domain d )\\
% \ib    InterfaceDescriptor \& interfaceDescriptor = interfaceList[inter]; \\
% \ib    \COM{Target and source grid functions:} \\
% \ib    GridFunction \& gfTarget = domainSolver[domainTarget]->gf[gfIndex[domainTarget]]; \\
% \ib    GridFunction \& gfSource = domainSolver[domainSource]->gf[gfIndex[domainSource]]; \\
% \ib    \\
% \ib    \COM{Get source data:} \\
% \ib    for( int face=0; face<gridListSource.size(); face++ ) \\
% \ic      domainSolver[domainSource]->interfaceRightHandSide( getInterfaceRightHandSide, ... ); \\
% \ib     \\
% \ib    \COM{Transfer the source arrays to the target arrays:} \\
% \ib    interfaceTransfer.transferData(... );  \\
% \ib    \\
% \ib    \COM{Adjust the target data before assigning:} \\
% \ib    for( int face=0; face<gridListTarget.size(); face++ ) \\
% \ic      \COM{Extrapolate the initial guess.} \\
% \ic      \COM{Under-relaxed iteration.} \\
% \ib    \\
% \ib    \COM{Assign the target data:} \\
% \ib    for( int face=0; face<gridListTarget.size(); face++ ) \\
% \ic      domainSolver[domainTarget]->interfaceRightHandSide( setInterfaceRightHandSide,...); \\
% \ia \\
% \}
% \end{flushleft}




% % ---------------------------------------------------------------------------------------------------------
% \clearpage
% \subsection{DomainSolver::interfaceRightHandSide} \label{sec:DomainSolver::interfaceRightHandSide}


% The {\tt DomainSolver::interfaceRightHandSide} function is used to get or set interface values.
% Each DomainSolver (cgad, cgcssi, cgins, cgsm,...) has a version of this routine. The generic
% version appears in cg/common/src/interfaceBoundaryConditions.C. 


% Here is {\tt Cgcssi::interfaceRightHandSide} (cg/cssi/src/interface.bC)
% \begin{flushleft}\tt\small
% Cgcssi::\FUNC{interfaceRightHandSide}( InterfaceOptionsEnum option, int interfaceDataOptions,  \\
%  \qquad\quad       GridFaceDescriptor \& info, GridFaceDescriptor \& gfd, int gfIndex, real t ) \\
% \{  \\
% \ia  RealArray \& bd = parameters.getBoundaryData(side,axis,grid,mg); \COM{Interface data on this domain.}\\
% \ia  RealArray \& f = *info.u; \COM{Interface data from another domain.} \\
% \ia \\
% \ia  \IF( interfaceType(side,axis,grid)==Parameters::heatFluxInterface ) \\
% \ib     \IF( option==setInterfaceRightHandSide ) \\
% \ic       bd(I1,I2,I3,tc)=f(I1,I2,I3,tc);  \COM{Set T (using values from another domain).}  \\
% \ib     \ELSEIF( option==getInterfaceRightHandSide )  \\
% \ic       \COM{Evaluate $a_0 T + a_1 T_n$ (to send to another domain).} \\
% \ic       f(I1,I2,I3,tc) = a[0]*uLocal(I1,I2,I3,tc) + a[1]*( normal(I1,I2,I2,0)*ux + ... ); \\
% \ia \\
% \ia  \ELSEIF( interfaceType(side,axis,grid)==Parameters::tractionInterface ) \\
% \ib     \IF( option==setInterfaceRightHandSide ) \\
% \ic       bd(I1,I2,I3,V)=f(I1,I2,I3,V);  \COM{Set positions of the interface.}  \\
% \ib     \ELSEIF( option==getInterfaceRightHandSide )  \\
% \ic       parameters.getNormalForce( gf[gfIndex].u,traction,ipar,rpar ); \\
% \ic       f(I1,I2,I3,V)=traction(I1,I2,I3,D); \\
% \ic       InterfaceDataHistory \& idh = gfd.interfaceDataHistory; \COM{Holds interface history.} \\
% \ic       \IF( interfaceDataOptions \& Parameters::tractionRateInterfaceData )\\
% \id         RealArray \& f0 = idh.interfaceDataList[prev].f;  \\
% \id         f(I1,I2,I3,Vt)= (f(I1,I2,I3,V) - f0(I1,I2,I3,V))/dt; \COM{Time derivative of the traction.} \\
% \ia   \\
% \}
% \end{flushleft}



% -------------------------------------------------------------------------------------------------
% \clearpage
% \bibliography{\homeHenshaw/papers/henshaw}
% \bibliographystyle{siam}


\end{document}
