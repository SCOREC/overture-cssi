\newcommand{\Gmc}{\Gc_{\rm mc}}% grid for multi-cyl random
\subsection{Shock hitting multiple rigid bodies}\label{sec:shockMultiCyl}

To run this example see {\tt cg/cssi/runs/multiCylRandom/Readme}.

% This example shows the simulation of a shock hitting a collection of cylinders that can move.
A shock moving from left to right impacts a collection of cylinders of random size. 
The fluid
forces cause the cylinders to move. The cylinders are modeled as rigid bodies with a given mass density.
A complex pattern of reflected and transmitted shocks results. Eventually some semblence of the 
incident shock appears to the right of the cylinders. 
This computation illustrates the use of moving grids and AMR. Collision
detection has been turned on.


%+ This simulation can be run with (multiCylRandomGrid.cmd is in Overture/sampleGrids)
%+ \begin{flushleft}\small\tt
%+   ogen -noplot multiCylRandomGrid -factor=4 -name=multiCylRandom4.order2.hdf \\
%+   cgcssi -noplot multiCylRandomMove -g=multiCylRandom2.order2.hdf -tf=1. -tp=.05 -show="multiCylRandom2.show" -go=go 
%+ \end{flushleft}
%+ Results can be plotted with
%+ \begin{flushleft}\small\tt
%+   plotStuff plotMultiCylRandom -show=multiCylRandom2.show
%+ \end{flushleft}


{
% 
\newcommand{\figWidth}{6.5cm}
\newcommand{\trimfig}[2]{\trimFigb{#1}{#2}{.05}{.05}{.25}{.25}}
% 
\begin{figure}[htb]
\begin{center}
\begin{tikzpicture}[scale=1]
  \useasboundingbox (0,.7) rectangle (14.,10);  % set the bounding box (so we have less surrounding white space)
%
 \draw ( 0, 5) node[anchor=south west,xshift=-4pt,yshift=+0pt] {\trimfig{\cssiDocDir/fig/multiCylRandom4Schlieren0p5}{\figWidth}};
 \draw (6.5,5) node[anchor=south west,xshift=-4pt,yshift=+0pt] {\trimfig{\cssiDocDir/fig/multiCylRandom4PressureAndAMR0p5}{\figWidth}};
%
 \draw ( 0, 0) node[anchor=south west,xshift=-4pt,yshift=+0pt] {\trimfig{\cssiDocDir/fig/multiCylRandom4Schlieren1p5}{\figWidth}};
 \draw (6.5,0) node[anchor=south west,xshift=-4pt,yshift=+0pt] {\trimfig{\cssiDocDir/fig/multiCylRandom4PressureAndAMR1p5}{\figWidth}};
%
% grid:
%   \draw[step=1cm,gray] (0,0) grid (14,10);
\end{tikzpicture}
\end{center}
  \caption{Shock hitting a collection of cylinders which move Schlieren images (left column) and pressure contours (right column)
 at times $t=0.5$ (top) and $t=1.5$ (bottom) on grid $\Gmc^{(4\times 2)}$. The block boundaries of the refinement
   grids are shown superimposed on the pressure contours.}% 
  \label{fig:shockMultiCylRandom}
\end{figure}
}


