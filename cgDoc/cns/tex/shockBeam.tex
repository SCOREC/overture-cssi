\subsection{Shock hitting a flexible beam}\label{sec:shockBeam}

To run these examples see {\tt cg/cssi/runs/beamInAChannel/Makefile}.

% The initial conditions consists of a circle or sphere of high pressure and density.

{
\begin{figure}[htb]
\begin{center}
\begin{tikzpicture}[scale=1]
  \useasboundingbox (0,.5) rectangle (16.,16.5);  % set the bounding box (so we have less surrounding white space)
%
  \begin{scope}[yshift=11cm]
    \figByHeightb{  0}{0}{fig/shockBeamColourSchlierenT0p7}{5.5cm}[.1][.2][.23][.24]; 
    \figByHeightb{  8}{0}{fig/shockBeamColourSchlierenT1p3}{5.5cm}[.1][.2][.23][.24];
  \end{scope} 
  \begin{scope}[yshift=5.5cm]
    \figByHeightb{  0}{0}{fig/shockBeamSchlierenT0p7}{5.5cm}[.1][.2][.23][.24]; 
    \figByHeightb{  8}{0}{fig/shockBeamSchlierenT1p3}{5.5cm}[.1][.2][.23][.24];
  \end{scope}  
  \begin{scope}[yshift=0cm]
    \figByHeightb{  0}{0}{fig/shockBeamPressureT0p7}{5.5cm}[.1][.2][.23][.24]; 
    \figByHeightb{  8}{0}{fig/shockBeamPressureT1p3}{5.5cm}[.1][.2][.23][.24];
  \end{scope}     
%
% grid:
   % \draw[step=1cm,gray] (0,0) grid (15,6.);
\end{tikzpicture}
\end{center}
  \caption{Shock hitting a deforming beam. Top: colour schlieren. Middle: schlieren. Bottom: pressure.}
  \label{fig:shockBeam}
\end{figure}
}