\section{Installing Overture on Ubuntu}  \label{sec:installUbuntu}


Here are instructions for Ubuntu version 18.04.

\bigskip\noindent
(I) Sudo install the following to get Ubuntu setup.
\begin{verbatim}
  apt install build-essential
  apt install build-essential
  apt install gfortran
  apt install autoconf
  apt install mpich
  apt install emacs    (optional)
  apt install libperl-dev
\end{verbatim}

\bigskip\noindent
(IIa) If you do not build your own version of OpenGL Mesa (instructions below),  then 
install OpenGL: (libjpeg62-dev may be needed even if you do install Mesa)
\begin{verbatim}
  apt install libmotif-dev
  apt install libgl1-mesa-dev
  apt install libglu1-mesa libxi-dev libxmu-dev libglu1-mesa-dev
  apt install libjpeg62-dev
\end{verbatim}

\noindent
(IIb) Optionally install Mesa OpenGL for better hardcopies.
Get MesaLib-7.2.tar.gz from OvertureFramework install page
\begin{verbatim}
  tar xzf MesaLib-7.2.tar.gz
  tar xzf MesaDemos-7.2.tar.gz
  cd Mesa-7.2/
  make linux-x86-64   (note: Error building demos, but lib's built OK) 

  --- REPLACE THE GL Header files in /usr/include with the MESA versions
  su -c "cp /home/henshw/software/Mesa-7.2/include/GLView.h /usr/include/GLView.h"
  su -c "cp -r /home/henshw/software/Mesa-7.2/include/GL /usr/include/GL"  
\end{verbatim}



\bigskip\noindent
(III) Install HDF (serial version) from hdf5-1.8.15-patch1.tar.gz -- follow normal instructions:
\begin{verbatim}
  tar xzf hdf5-1.8.15-patch1.tar.gz
  mv hdf5-1.8.15-patch1 hdf5-1.8.15-gcc7.4.0

  cd hdf5-1.8.15-gcc7.4.0
  unsetenv cc
  setenv CC gcc
 ./configure --prefix=`pwd` 
  make
  make install
\end{verbatim}

\bigskip\noindent
(IV) Install A++ following normal instructions (get AP-0.8.2.tar.gz from overtureFramework.org)
\begin{verbatim}
  su -c "ln -s /usr/bin/make /usr/bin/gmake"

  tar -xzf AP-0.8.2.tar.gz
  mv A++P++-0.8.2 AppPpp-0.8.2-gcc7.4.0
  cd AppPpp-0.8.2-gcc7.4.0
  configure --enable-SHARED_LIBS --prefix=`pwd`
  make -j24
  make install
\end{verbatim}

\bigskip\noindent
(V) Optionally install PETSc 3.4.5 (linear solvers for cgins, for example)
\begin{verbatim}
  tar xzf petsc-3.4.5.tar.gz
  mv petsc-3.4.5 petsc-3.4.5-serial
  setenv PETSC_DIR /home/henshw/software/petsc-3.4.5-serial
  cd petsc-3.4.5-serial
  configure --with-cc=gcc --with-fc=gfortran --PETSC_ARCH=linux-gnu-opt --with-debugging=0 \
            --with-fortran=0 --with-matlab=0 --with-mpi=0 --with-shared-libraries=1
  make 
\end{verbatim}


\bigskip\noindent
(VI) Install Overture from sourceforge: -- if you have an account on sourceforge you can download the
latest Overture using 
\begin{verbatim}
   git clone ssh://USERNAME@git.code.sf.net/p/overtureframework/code /home/USERNAME/overtureFramework
\end{verbatim}

Goto overtureFramework/Overture and read the README file there.
\begin{verbatim}
   cd /home/USERNAME/overtureFramework/Overture
   # Location for compiled version of Overture g=serial debug version : 
   setenv OvertureBuild /home/USERNAME/Overture.g   
   buildOverture
   cd /home/USERNAME/Overture.g 
   -- set Overture enviromental variables found in Overture.g/defenv  ---
   configure
   make -j8    
\end{verbatim}

To build cg:
\begin{verbatim}
   setenv CG /home/USERNAME/overtureFramework/cg  (location of cg source files)
   setenv CGBUILDPREFIX /home/USERNAME/cg.g       (location for compiled version of cg)
   cd /home/USERNAME/overtureFramework/cg
   make   
\end{verbatim}

\bigskip 
\bigskip 
\bigskip\noindent
For the \textbf{parallel version}, install P++ 
\begin{verbatim}
  tar xzf AP-0.8.2.tar.gz
  mv A++P++-0.8.2 AppPpp-0.8.2-gcc7.4.0-parallel
  
  cd AppPpp-0.8.2-gcc7.4.0-parallel
  NOTE: mpi include directory: 
  configure --enable-PXX --prefix=`pwd` --enable-SHARED_LIBS \
            --with-mpi-include=-I${MPI_ROOT}/include/mpich \  
            --with-mpi-lib-dirs="-L${MPI_ROOT}/lib" --with-mpi-libs="-lmpich -lmpl -lc -lm" \
            --with-mpirun=${MPI_ROOT}/bin/mpiexec --without-PADRE --disable-mpirun-check 
  make -j24
  make install
\end{verbatim}

\bigskip\noindent
Parallel hdf5:
\begin{verbatim}
  tar xzf hdf5-1.8.15-patch1.tar.gz
  mv hdf5-1.8.15-patch1 hdf5-1.8.15-gcc7.4.0-parallel

  cd hdf5-1.8.15-gcc7.4.0-parallel
  unsetenv CC
  unsetenv cc
  setenv CC $MPI_ROOT/bin/mpicc
   ./configure --prefix=`pwd` --enable-parallel
  make
  make install
\end{verbatim}

\bigskip\noindent
To build parallel Overture
\begin{verbatim}
   cd /home/USERNAME/overtureFramework/Overture
   # Location for compiled version of Overture p=parallel debug version : 
   setenv OvertureBuild /home/USERNAME/Overture.p    
   buildOverture
   cd /home/USERNAME/Overture.p 
   -- set Overture enviromental variables found in Overture.g/defenv  ---
   configure parallel
   make -j8    
\end{verbatim}

\bigskip\noindent
To build parallel cg:
\begin{verbatim}
   setenv CG /home/USERNAME/overtureFramework/cg  (location of cg source files)
   setenv CGBUILDPREFIX /home/USERNAME/cg.p       (location for compiled version of cg)
   cd /home/USERNAME/overtureFramework/cg
   make   
\end{verbatim}


% 
% \begin{verbatim}
% Sudo intall the following Ubuntu packages:
% -- build-essential
% -- gfortran
% -- autoconf
% -- mpich (this one is optional)
% 
% -- tcsh
% -- emacs
% 
% -- libmotif-dev
% -- libgl1-mesa-dev (DIR: x86_64-linux-gnu) this part is taken care of by supplying the distribution
% -- libperl-dev
% -- libglu1-mesa libxi-dev libxmu-dev libglu1-mesa-dev
% -- libjpeg62-dev
% 
% **Note: 
% ***in file config/MakeDefs.linux, remove "-lXp", possibly "-lXmu -lXext" too 
% ***in ".cshrc", change LOCALE to en-US.UTF-8 
% 
% -- hdf5
% **Note: line 647, h5diff.c, change C++ comment "//" to "/* */"
% 
% -- A++
% **Note in the /usr/bin directory: sudo ln -s make gmake (need gmake)
% 
% -- LAPACK BLAS
% libblas-dev liblapack-dev (install DIR: x86_64-linux-gnu) this part is taken care of by supplying the distribution
% 
% -- Overture
% ./configure distribution=ubuntu
% **Note: getAnswer.C line 2107, change '\0' to NULL
% **Note: set MOTIF,XLIBS,OPENGL,LAPACK to /usr and the rest of the environment to the install dir.
% 
% \end{verbatim}
% 
% 
% 
% 
% \subsection{OS X 10.9 - Mavericks} 
% 
% \vskip\baselineskip
% \noindent{\bf Stage I:}
% The default compiler for Mavericks is {\em clang} which replaces gcc and g++. DO NOT install gcc. 
% \begin{enumerate}
%   \item Install Xcode from the apple web site.
%   \item Install the command line tools in Xcode: 
%     \begin{verbatim}
%        xcode-select --install
%     \end{verbatim}
%     This command will ask you whether you want to install the command lines tools. You will know if you have the
%     command lines tools if you can issue the command {\tt gcc --version}. 
%   \item Install gfortran from {\tt http://gcc.gnu.org/wiki/GFortranBinaries\#MacOS}
%   \item Install XQuartz (if you do not already have it). This should install X11 libraries in {\tt /opt/X11/lib}. This is needed
%                  so you can pop up X windows but since there is no Motif here we will need to install another version using Macports.
%   \item Install Macports
%   \item Use Macports to install Open Motif and OpenGL:
%      \begin{verbatim}
%          sudo port install openmotif
%          sudo port install mesa
%          sudo port install libGLU
%     \end{verbatim}
%     This will install another version of X11 in {\tt /opt/local/lib} in addition to the Open Motif and Mesa libraries. 
% \end{enumerate}
% 
