%-----------------------------------------------------------------------
%   Overture Documentation
%-----------------------------------------------------------------------
\documentclass{article}

% \setlength{\textwidth}{9in}      % page width

% \usepackage{verbatim}
% \usepackage{moreverb}
% \usepackage{graphics}    
% \usepackage{epsfig}    
% \usepackage{calc}
% \usepackage{ifthen}
% \usepackage{float}
% \usepackage{fancybox}

\usepackage{html}

\newcommand{\documentation}{../documentation}
\newcommand{\publications}{../publications}

\begin{document}

\begin{center}
{\Large Overture Documentation} \\
~~ \\
LLNL Overlapping Grid Project \\
\htmladdnormallink{Centre for Applied Scientific Computing}{http://www.llnl.gov/casc} \\
\htmladdnormallink{Lawrence Livermore National Laboratory}{http://www.llnl.gov}  \\
\htmladdnormallink{Overture}{http://www.llnl.gov/CASC/Overture} 
\end{center}


\input announcement.tex

\input mailingList.tex

% \input bugList.tex

{\bf Overture.v24} is now released. Here are the \htmladdnormallink{installation instructions}{../install/index.html}.


{\bf cg.v24} is now available. These are a suite of Overture based Composite-Grid PDE solvers.
Includes {\bf cgcssi} : compressible Navier-Stokes and
reactive-Euler equations, {\bf cgins} : incompressible Navier-Stokes with heat transfer, {\bf cgasf} : all-speed compressible
Navier-Stokes, {\bf cgmx} time-domain Maxwell's equations, {\bf cgad} advection-diffusion, 
{\bf cgsm} : elastic wave equation (solid mechanics),
{\bf cgmp} : a multi-physics solver.


% \htmladdnormallink{OvertureDoc.ps.v17.tar.gz}{\documentation/OvertureDoc.v17.ps.tar.gz} (8.7M) 
% contains {\bf all} the Overture documentation in postscript form.
% \htmladdnormallink{OvertureDoc.v17.tar.gz}{\documentation/OvertureDoc.v17.tar.gz} (12.3M) 
% contains {\bf all} the Overture documentation in TeX dvi form (plus all figures in postscript form)
% so that you can view the documentation with {\tt xdvi} or print it with {\tt dvips}.

Here are some demo and short course documents for Overture,
\begin{itemize}
  \item Overture Demo Introduction, \htmladdnormallink{OvertureDemoIntro.pdf}{\publications/OvertureDemoIntro.pdf}, K. Chand and W. Henshaw.
  \item Hands-on Demo of Overture and the CG Solvers, \htmladdnormallink{OvertureDemo.pdf}{\publications/OvertureDemo.pdf}, K. Chand and W. Henshaw.
  \item Short Course: Solving PDE's on Overlapping Grids with the Overture Framework,
          \htmladdnormallink{OvertureShortCourse.pdf}{\publications/OvertureShortCourse.pdf}, W. Henshaw (updated May 2011).
\end{itemize}


Here are the separate documents related to the Overture C++ classes for solving PDEs. A good place to start
is with the {\bf primer} documentation.

Here are some documents that give one an idea of the capabilities of Overture
\begin{itemize}
  \item A Primer for Overture, 
        \htmladdnormallink{primer.pdf}{\documentation/primer.pdf}, 
        % \htmladdnormallink{html version}{\documentation/primerHTML/primerHTML.html}, 
        by W. Henshaw. This document presents
        a series of C++ programs that use the classes in Overture to solve PDEs. {\bf This is a good place to start}.
  \item An overlapping grid generator, {\bf ogen}, 
        \htmladdnormallink{ogen.pdf}{\documentation/ogen.pdf},
        % \htmladdnormallink{html version}{\documentation/ogenHTML/ogenHTML.html},
        by W. Henshaw. Includes extensive
        capabilities for generating component grids together with an automatic algorithm for computing
        the overlap.
  \item An unstructured (hybrid) grid generator, {\bf ugen}, 
        \htmladdnormallink{unstructured.pdf}{\documentation/unstructured.pdf},
        % \htmladdnormallink{html version}{\documentation/unstructuredHTML/unstructuredHTML.html},
        by Kyle Chand. Also by Kyle, a 2D unstructured grid generator, {\bf smesh}, 
        \htmladdnormallink{smesh.pdf}{\documentation/smesh.pdf}.
 \item {\bf cg}, a suite of Composite-Grid PDE solvers (replacing OverBlown). Documentation is under development. 
    \begin{itemize}
       \item Cgins: (Incompressible Navier-Stokes), 
               User Guide~\htmladdnormallink{CginsUserGuide.pdf}{\documentation/CginsUserGuide.pdf},
               and Reference manual~\htmladdnormallink{CginsRef.pdf}{\documentation/CginsRef.pdf}.
       \item Cgcssi: See the OverBlown documentation for the time being. 
    \end{itemize}
 \item {\bf OverBlown}, ({\bf now replaced by cg}) a flow solver for the Navier-Stokes equations, including algorithms for incompressible,
        compressible and slightly compressible flows. User Guide~\htmladdnormallink{ob.pdf}{\documentation/ob.pdf}, 
        and Reference Manual~\htmladdnormallink{obRef.pdf}{\documentation/obRef.pdf}.
\end{itemize}

Here are documents describing A++ and P++, the serial and parallel array classes
\begin{itemize}
  \item A++/P++ Manual,\htmladdnormallink{manual.ps.gz}{\documentation/App/manual.ps.gz},
        \htmladdnormallink{html version}{\documentation/App/manual/manual.html}
  \item A++/P++ Quick Reference Manual, 
     \htmladdnormallink{Quick\_Reference\_Manual.ps.gz}{\documentation/App/Quick\_Reference\_Manual.ps.gz}
     \htmladdnormallink{html version}{\documentation/App/Quick\_Reference\_Manual/Quick_Reference_Manual.html},
  \item A++/P++ Tutorial Guide, 
       \htmladdnormallink{Tutorial\_Guide.ps.gz}{\documentation/App/Tutorial_Guide.ps.gz},
       \htmladdnormallink{html version}{\documentation/App/Tutorial_Guide/Tutorial_Guide.html}
  \item A++/P++ Reference Examples, 
          \htmladdnormallink{Reference\_Examples.ps.gz}{\documentation/App/Reference_Examples.ps.gz},
          \htmladdnormallink{html version}{\documentation/App/Reference\_Examples/Reference\_Examples.html}
\end{itemize}

Here are more detailed documents needed when developing software within the Overture framework
\begin{itemize}
  \item The Overture developers guide 
    \htmladdnormallink{developersGuide.pdf}{\documentation/developersGuide.pdf},
     % \htmladdnormallink{html version}{\documentation/developersGuideHTML/developersGuideHTML.html},
    by W. Henshaw gives some useful info for those writing software with Overture. This is mainly intended
    for developers on the Overture team.
  \item Here is a master index for the Overture documentation 
     \htmladdnormallink{masterIndex.pdf}{\documentation/masterIndex.pdf},
     % \htmladdnormallink{html version}{\documentation/masterIndex/masterIndex.html}.
      This is version is far from complete, just a first try.
  \item The Mapping classes for describing geometry, including spheres, cylinders, splines, NURBS,
         bodies of revolution, elliptic and hyperbolic grid generation etc.
        \htmladdnormallink{mapping.pdf}{\documentation/mapping.pdf},
        % \htmladdnormallink{html version}{\documentation/mappingHTML/mappingHTML.html},
        \htmladdnormallink{hyperbolic.pdf}{\documentation/hyperbolic.pdf}, 
        % \htmladdnormallink{html version}{\documentation/hyperbolicHTML/hyperbolicHTML.html},
        by W. Henshaw. You will want
        to read this if you are generating grids with {\bf ogen}.
 \item The Grid Classes, user guide
        \htmladdnormallink{gridGuide.pdf}{\documentation/gridGuide.pdf},
        % \htmladdnormallink{html version}{\documentation/gridGuideHTML/gridGuideHTML.html} 
        and reference manual \htmladdnormallink{grid.pdf}{\documentation/grid.pdf}, 
        % \htmladdnormallink{html version}{\documentation/gridHTML/gridHTML.html} and 
         by G. Chesshire and W. Henshaw.
  \item Grids Functions,
        \htmladdnormallink{gf.pdf}{\documentation/gf.pdf}, 
        % \htmladdnormallink{html version}{\documentation/gfHTML/gfHTML.html},
         by W. Henshaw. Grid functions are used
         in Overture to represent field variables such as density, velocity, pressure.
  \item Other Stuff: utility functions for Overture,
        \htmladdnormallink{otherStuff.pdf}{\documentation/otherStuff.pdf},
        % \htmladdnormallink{html version}{\documentation/otherStuffHTML/otherStuffHTML.html},
         by W. Henshaw. Useful
         support stuff when writing Overture codes.
  \item Difference Operators and Boundary Conditions,
        \htmladdnormallink{op.pdf}{\documentation/op.pdf},
        % \htmladdnormallink{html version}{\documentation/opHTML/opHTML.html},
       by W. Henshaw. Second and fourth order,
       conservative and non-conservative, difference approximations to differential operators and a
       large collection of elementary boundary conditions.
%    \item Finite volume Operators for Overture
%    \htmladdnormallink{MGFVO.ps.gz}{\documentation/MGFVO.ps.gz}, UCRL-MA-133649, by D. Brown.
%          {\bf NOTE:} These finite volume operators will probably disappear in a future release. Use
%          the conservative finite difference operators instead.
  \item Overture Sparse System Solver,
        \htmladdnormallink{oges.pdf}{\documentation/oges.pdf}, 
        % \htmladdnormallink{html version}{\documentation/ogesHTML/ogesHTML.html},
        by P. Fast and W. Henshaw. This class acts as an
         interface from Overture coefficient matrices to various solvers for sparse matrices.
  \item Plotting Overture objects with OpenGL,
        \htmladdnormallink{GraphicsDoc.pdf}{\documentation/GraphicsDoc.pdf},
        % \htmladdnormallink{html version}{\documentation/GraphicsDocHTML/GraphicsDocHTML.html},
        by W. Henshaw and A. Petersson.
%  \item Plotting Overture objects with OpenGL,
%        \htmladdnormallink{PlotStuff.pdf}{\documentation/PlotStuff.pdf}, 
%        \htmladdnormallink{PlotStuff.pdf}{\documentation/PlotStuff.pdf},
%        \htmladdnormallink{html version}{\documentation/PlotStuffHTML/PlotStuffHTML.html},
%        by W. Henshaw and A. Petersson.
  \item The plotStuff graphics post-processor for Overture, used to plot results saved in show files,
        \htmladdnormallink{plotStuff.pdf}{\documentation/plotStuff.pdf},
        % \htmladdnormallink{html version}{\documentation/plotStuffHTML/plotStuffHTML.html},
         by W. Henshaw.
  \item Saving results from a PDE solver to be plotted later,
        \htmladdnormallink{ogshow.pdf}{\documentation/ogshow.pdf}, 
        % \htmladdnormallink{html version}{\documentation/ogshowHTML/ogshowHTML.html},
        by W. Henshaw.
  \item An interface to scientific data-bases and an implementation based on HDF,
        \htmladdnormallink{db.pdf}{\documentation/db.pdf}, 
        % \htmladdnormallink{html version}{\documentation/dbHTML/dbHTML.html},
        by W. Henshaw. This is the generic interface
        used by Overture in order to interact with a data-base file.
  \item Ogmg: a multigrid solver for overlapping grids, 
         \htmladdnormallink{ogmg.pdf}{\documentation/ogmg.pdf},
        % \htmladdnormallink{html version}{\documentation/ogmgHTML/ogmgHTML.html},
          by W. Henshaw.  
  \item Adaptive Mesh Refinement Routines for Overture, \htmladdnormallink{amr.pdf}{\documentation/amr.pdf},
        % \htmladdnormallink{html version}{\documentation/amrHTML/amrHTML.html},
          by D. Brown and W. Henshaw.
  \item Liner Builder: A Tool for Building Geometric Models of Shaped Charge Liners, 
         \htmladdnormallink{linerBuilder.pdf}{\documentation/linerBuilder.pdf},
          by W. Henshaw.
%  \item A Stencil class that can be used to generate arbitrary stencil operations for grid functions,
%        \htmladdnormallink{Stencil.ps}{\documentation/Stencil.ps}, by Krister \AA hlander.
\end{itemize}

Here are other related documents
\begin{itemize}
  \item {\em Time Step Determination for PDEs with Applications to Programs Written with Overture} 
   \htmladdnormallink{timeStep.pdf}{\documentation/timeStep.pdf}. Shows one how to compute the time
    step for a PDE discretized on a curvilinear grid with Overture operators.
% \item AMR++ design document \htmladdnormallink{AMR++}{\documentation/AMR++.ps.gz}, by D. Quinlan. Describes
%    the adapative mesh refinement classes used by Overture.
\end{itemize}

% Here is documentation on projects that use Overture 
% \begin{itemize}
%  \item The Cogito/Overhere problem solving enviroment by Krister \AA hlander, 
%    \htmladdnormallink{Overhere.ps}{http://www.c3.lanl.gov/\%7Ekrister/Overhere/Overhere.ps}.
% \end{itemize}



% \bibliography{/users/henshaw/papers/henshaw}
% \bibliographystyle{siam}
\end{document}
